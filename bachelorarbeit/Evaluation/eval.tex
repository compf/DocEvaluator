\newcommand{\checkpmd}{\textit{Checkstyle} und \textit{PMD} }
\newcommand{\doceval}{\textit{DocEvaluator} }
In diesem Kapitel soll das Programm evaluiert werden, um zu prüfen, ob das Programm (im Folgenden \textit{DocEvaluator}) eine gute heuristische Aussage über den Stand der Dokumentationsqualität treffen kann. Dazu wird der \textit{DocEvaluator} mit \checkpmd{} verglichen, indem drei exemplarische Open-Source-Projekte mit allen drei Programmen analysiert werden und die Treffergenauigkeit und Geschwindigkeit der Programme verglichen werden. 

\section{Wahl der zu analysierenden Projekte}
Zur Durchführung der Evaluation wurden verschiedene Softwareprojekte aus GitHub heruntergeladen. Grundsätzlich kann der Vergleich mit jedem Java-Projekt durchgeführt werden, dennoch wurden einige Bedingungen festgelegt, die bei der Auswahl der Projekte eine wichtige Rolle spielen. Diese Bedingungen werden in der folgenden Auflistung präsentiert:

\begin{enumerate}
    \item \label{enum:size} Die Projekte müssen mindestens einen Umfang von 10\,000 \ac{LOC} haben
    \item \label{enum:already_cited} Die Projekte müssen bereits in einer in dieser Bachelorarbeit zitierten Quelle in puncto Dokumentationsqualität analysiert worden sein
    \item \label{enum:parsing_error}  Die Projekte sollen möglichst wenige Parsing-Fehler beim \doceval produzieren. Bei mehr als als zwei Fehlern wird ein Projekt nicht betrachtet
    \item Mindestens ein Projekt sollte deutlich größer sein als die anderen Projekte
\end{enumerate}
 Durch die Bedingung in Nr. \ref{enum:size} wird sichergestellt, dass eine ausreichende Anzahl an Fehlern in der Dokumentation zu erwarten ist, um eine gute Analyse der Dokumentationsqualität  und eine aussagekräftige Bewertung der Geschwindigkeit zu ermöglichen. Durch die Bedingung in Nr.  \ref{enum:already_cited} werden nur Projekte in Betracht gezogen, die bereits in der wissenschaftlichen Literatur berücksichtigt wurden und daher (zumindest für den jeweiligen Autor der Quelle) geeignet für eine Analyse der Dokumentation sind. Mit der Bedingung in Nr. \ref{enum:parsing_error} wird eine Verzerrung zugunsten oder zuungunsten des \textit{DocEvaluators} vermieden, denn durch Parsing-Fehler können Komponenten falsch analysiert werden, die von den anderen Tools richtig analysiert werden. Indirekt wird dadurch gefordert, dass ein Projekt keine neueren Java-Funktionen verwendet, da  der \textit{DocEvaluator} nur Java bis Version 8 unterstützt. Die letzte Bedingung ist besonders für die Bewertung der Geschwindigkeit relevant, da so geprüft werden kann, ob der \doceval auch bei größeren Projekten noch in annehmbarer Zeit ein Ergebnis berechnen kann. 
 
 \subsubsection{Analysierte Projekte}
 Die folgende Auflistung zeigt die gewählten Projekte für die Evaluation. In Klammern dahiner befinden sich die  (gerundeten) \ac{LOC}
 \begin{itemize}
    \item ArgoUML (170.000)(
     \item Log4J (20.000)
     \item Eclipse \ac{JDT} (400.000)
 \end{itemize}
 
 ArgoUML und Eclipse wurden in \cite[S. 74] {AutomaticQualityAssessmentofSourceCodeComments:TheJavadocMiner} bewertet, wobei hier nur Eclipse \ac{JDT} betrachtet werden soll, da das gesamte Eclipse-Projekt zu umfassend ist. Log4J wurde in \cite[S. 267] {@tComment:TestingJavadocCommentstoDetectComment-CodeInconsistencies} betrachtet. 

 Bei allen Projekten traten zunächst Parsing-Fehler auf. Dies lag allerdings in den meisten Fällen daran, dass auskommentierte Methoden noch einen Javadoc-Präfix hatten. Dies kann vom \doceval nicht korrekt verarbeitet werden. Da diese Methoden offensichtlich nicht verwendet werden und keinen Einfluss auf die Qualität der Dokumentation haben können, wurden sie ersatzlos entfernt. Bei einen Fehler in Eclipse JDT konnte der \doceval eine Datei mit mehr als 3000 Codezeilen nicht verarbeiten. Diese Datei wurde bei der Evaluation ignoriert.
\section{Analyse der Qualität}
Zur Durchführung der Evaluation muss zunächst definiert werden, welche Funktionen der einzelnen Programme miteinander verglichen werden können, da die Programme unterschiedliche Aspekte der Dokumentation überprüfen und die Darstellung der Ergebnisse im Vergleich zum \textit{DocEvaluator} abweicht.

\checkpmd können als fehlersuchend bezeichnet werden. Sie prüfen die einzelnen Komponenten eines Programms und finden Abweichungen von vorher definierten Regeln. Eine solche Regel kann beispielsweise sein, dass jede öffentliche Methode dokumentiert sein muss, dass bestimmte Wörter nicht verwendet werden dürfen oder dass die Syntax der Dokumentation Fehler enthält. Damit sind sie vergleichbar mit den Metriken aus den Kapiteln \ref{chapter:metrics_coverage}  und \ref{chapter:metrics_errors}, welche ebenfalls bestimmte Fehler suchen und bei einem Verstoß gegen die Regeln eine Warnmeldung ausgeben. Allerdings berechnen \checkpmd keine Metriken, sondern finden nur die besagten Verstöße gegen die definierten Regeln. Somit kann ein Entwickler sehen, dasś ein Projekt beispielsweise 100 Verstöße gegen die Dokumentationsrichtlinien hat, erfährt aber nicht, ob die Anzahl der Verstöße unter Berücksichtigung der Projektgröße schwerwiegend sind und erhält keine normierte Bewertung, die dem Entwickler bei der Beurteilung der Dokumentationsqualität hilft. 

Im Gegensatz dazu verwendet der \doceval Metriken, die stets einen Wert von 0 bis 100 zurückgeben, sodass ein Entwickler weiß, dass ein hoher Wert für eine hohe Qualität steht. Außerdem kann der \doceval auch die Semantik des Kommentars heuristisch prüfen, um zu erfahren, ob der Kommentar verständlich ist und nicht redundant ist (vgl. Kapitel \ref{chapter:metrics_semantic}). Nichtsdestotrotz gibt der \doceval auch Warnmeldungen aus, wenn er bestimmte Komponenten schlechter bewerten muss.

Aus diesen Gründen wird die Evaluation nicht mit den Metriken an sich, sondern mit den Warnmeldungen der jeweiligen Tools durchgeführt. Jedes Programm wird mit einem Verweis auf das zu analysierende Projekt aufgerufen und die Ausgaben der Programme werden in separaten Dateien umgeleitet.  Diese Dateien dienen dann als Grundlage für die spätere Evaluation. 

\subsubsection{Durchführung der Evaluation}
Wie im vorherigen Absatz beschrieben, dienen die Logdateien der drei Programme als Datengrundlage für die Evaluation. Jede Zeile in diesen Logdateien enthält mindestens den Dateipfad des gefundenen Fehlers, die Zeilennummer des Fehlers und einen Fehlercode als Zeichenkette.  Wenn die drei Programme einen übereinstimmenden Fehler finden, sollte es in allen drei Logdateien einen Eintrag geben, bei dem der Dateipfad identisch ist, die Zeilennummer innerhalb eines gewissen Intervalls identisch ist und die der Fehlercode identisch ist. Die Zeilennummer muss nicht zwingen identisch sein, da  \checkpmd die exakte Zeile des Fehlers ausgeben, sodass beispielsweise bei einem unzulässigen Wort die exakte Zeile des Wortes genannt wird, während der \doceval nur die Zeile der betroffenen Komponente ausgibt. Da die drei Tools unterschiedliche Fehlercodes ausgeben, werden diese so kategorisiert, dass Verstöße, welche von mehr als einem Tool erkannt werden, einen eigenen Fehlercode erhalten. Alle anderen Verstöße erhalten einen allgemeinen programmspezifischen Fehlercode, der somit keine näheren Informationen über die Art des Fehlers hergibt.

Basierend auf diesen Vorbereitungen kann nun für jeden Fehler jedes Programm gefunden werden, das diesen Fehler gefunden kann. Beispielsweise kann ein Fehler von allen drei Programmen, nur von \textit{Checkstyle} oder nur von \textit{PMD} und \doceval gefunden werden. Mathematisch gesehen kann die Potenzmenge der Menge \textit{\{Checkstyle, PMD, DocEvaluator\}} genommen werden, wodurch alle möglichen Kombinationen an Programmen entstehen, die ein bestimmten Fehler erkennen können. Durch Zählen der Fehler pro Programmkombination lässt sich bewerten, welche Programme besonders viele Fehler finden, die andere Programme nicht erkennen und ob alle drei Programme viele identische Fehler finden. 


\subsubsection{Auswahl der Regeln}
Zum Vergleich der drei Programme müssen zunächst Regeln festgelegt werden, bei denen die Programme eine Warnmeldung ausgeben. Bei \checkpmd{} erfolgt die Konfiguration über  Extensible Markup Language-Dateien. Beim \doceval erfolgt die Konfiguration über das in Kapitel \ref{chapter:conf} beschriebene \ac{JSON}-Format. Bei der Auswahl der Regeln muss beachtet werden, dass \checkpmd auch andere Fehler wie z.~B. komplexe Methoden finden können. Diese sind in diesem Kontext nicht relevant und werden ignoriert. Außerdem können die drei Programme zum Teil unterschiedliche Fehler finden, da beispielsweise \textit{PMD} (anders als \textit{Checkstyle}) den Inhalt der Dokumentation auf das Auftauchen bestimmter Begriffe prüfen kann, \textit{Checkstyle} aber dafür (anders als \textit{PMD}) prüfen kann, ob die Block-Tags in der richtigen Reihenfolge definiert werden. Die letztere Regel wird vom \doceval in der ausgelieferten Fassung nicht geprüft. Alle drei Programme können jedoch prüfen, ob eine Komponente dokumentiert ist oder nicht. Tabelle \ref{tab:inters_rules} vergleicht die Überschneidungen der Regeln, welche die drei Programme anwenden können. Dabei stehen die Abkürzungen \textit{CS} für \textit{Checkstyle} und \textit{DE} für \textit{DocEvaluator}:

\begin{table}[]
    \centering
    \begin{tabular}{m{4.5cm}|m{4.5cm}|m{4.5cm}}
     \textbf{CS} $\cap$ \textbf{DE}  & \textbf{PMD} $\cap$ \textbf{DE} & \textbf{PMD} $\cap$ \textbf{DE} $\cap$  \textbf{CS}  \\\hline
     \begin{itemize}
        \item Komponente dokumentiert
        \item Methode vollständig dokumentiert
         \item Fehler in Javadoc
     \end{itemize}
      & 
      \begin{itemize}
          \item  Komponente dokumentiert

          \item Bestimmte Wörter in Kommentar verbieten
      \end{itemize}
      & 
       \begin{itemize}
          \item  Komponente dokumentiert
         
      \end{itemize}
      \\\hline
    \end{tabular}
    \caption{Überschneidungen der Regeln der drei Programme}
    \label{tab:inters_rules}
\end{table}

Aus der Tabelle lässt sich entnehmen, dass der \doceval mit \textit{Checkstyle} die meisten Übereinstimmungen hat. Beide Programme können die Vollständigkeit der Methodendokumentation und typische Fehler in Javadoc-Kommentaren (wie z.~B. fehlerhaftes HTML) finden. PMD und der \doceval können bestimmte Wörter in der Dokumentation bemängeln, die in einer Dokumentation vermieden werden sollten. Alle drei Programme können das Vorhandensein der Dokumentation überprüfen. Allerdings ignoriert \textit{PMD} alle privaten Komponenten außer Felder. Vor allem bei Checkstyle gibt es zudem viele Regeln, die von den anderen beiden Programmen nicht gefunden werden und auf der Webseite \cite{checkstyle_doc_metrics} aufgelistet werden.

Da ein Vergleich der drei Tools somit nur eingeschränkt möglich ist, wird sich die qualitative Evaluation nur auf den Kernbereich beschränken. Es werden also nur die Regeln verwendet, die von mindestens zwei Tools unterstützt werden. Da somit nur relativ leichte Fehler gefunden werden, können die Ergebnisse der Evaluation dazu verwendet werden, um die Qualität des Parsings zu ermitteln, denn wenn solche grundlegenden Fehler (wie z.~B. das Nichtvorhandensein der Dokumentation) nicht gefunden werden, besteht eine erhebliche Chance, dass eine Java-Datei falsch interpretiert wird. 



\subsection{Ergebnisse}


\begin{table}[]
    \centering
    \begin{tabular}{c|c|c|c}
    Name & ArgoUml & Log4J & Eclipse JDT\\\hline
    |DE| & 10.054 & 1.710 & 17.380\\\hline
    |CS| & 9.961 & 1.590 & 17.638\\\hline
    |PMD|& 9.051 & 1.008 & 12.702\\\hline
    
    \{DE\} & 124 & 108 & 555\\\hline
    \{CS\} & 285 & 26 & 273\\\hline
    \{PMD\} & 377 & 86 & 298\\\hline
    \{PMD, DE\} & 264 & 41 & 253\\\hline
    \{CS, DE\} & 1266 & 683 & 5.214\\\hline
    \{PMD, CS\} & 10 & 3 & 793\\\hline
    \{PMD, CS, DE\} & 8.400 & 878 & 11.358\\\hline
    \end{tabular}
    \caption{Ergebnisse der Evaluation}
    \label{tab:eval_results}
\end{table}
