\newcommand{\checkpmd}{\textit{Checkstyle} und \textit{PMD} }
\newcommand{\doceval}{\textit{DocEvaluator} }
In diesem Kapitel soll das Programm evaluiert werden, um zu prüfen, ob das Programm (im Folgenden \textit{DocEvaluator}) den Anforderungen genügt.  Zunächst wird der \textit{DocEvaluator} mit \checkpmd{} verglichen, indem drei exemplarische Open-Source-Projekte mit allen drei Programmen analysiert werden und die Treffergenauigkeit und Geschwindigkeit der Programme verglichen werden. Außerdem wird  erläutert, welche Unit-Tests zur Verifikation des Programmverhaltens eingesetzt wurden.
\section{Vorgehensweise bei der Evaluation}
Insgesamt soll die Evaluation zwei Aspekte überprüfen. Zunächst soll die Qualität der drei Programme geprüft werden. Anschließend soll die Geschwindigkeit der Programme analysiert werden.

\subsection{Analyse der Qualität}
Zur Durchführung der Evaluation muss zunächst definiert werden, welche Funktionen der einzelnen Programme miteinander verglichen werden können, da die Programme unterschiedliche Aspekte der Dokumentation überprüfen und die Darstellung der Ergebnisse im Vergleich zum \textit{DocEvaluator} abweicht.

\checkpmd können als fehlersuchend bezeichnet werden. Sie prüfen die einzelnen Komponenten eines Programms und finden Abweichungen von vorher definierten Regeln. Eine solche Regel kann beispielsweise sein, dass jede öffentliche Methode dokumentiert sein muss, dass bestimmte Wörter nicht verwendet werden dürfen oder dass die Syntax der Dokumentation Fehler enthält. Damit sind sie vergleichbar mit den Metriken aus den Kapiteln \ref{chapter:metrics_coverage}  und \ref{chapter:metrics_errors}, welche ebenfalls bestimmte Fehler suchen und bei einem Verstoß gegen die Regeln eine Warnmeldung ausgeben. Allerdings berechnen \checkpmd keine Metriken, sondern finden nur die besagten Verstöße gegen die definierten Regeln. Somit kann ein Entwickler sehen, dasś ein Projekt beispielsweise 100 Verstöße gegen die Dokumentationsrichtlinien hat, erfährt aber nicht, ob die Anzahl der Verstöße unter Berücksichtigung der Projektgröße schwerwiegend sind und erhält keine normierte Bewertung, die dem Entwickler bei der Beurteilung der Dokumentationsqualität hilft. 

Im Gegensatz dazu verwendet der \doceval Metriken, die stets einen Wert von 0 bis 100 zurückgeben, sodass ein Entwickler weiß, dass ein hoher Wert für eine hohe Qualität steht. Außerdem kann der \doceval auch die Semantik des Kommentars heuristisch prüfen, um zu erfahren, ob der Kommentar verständlich ist und nicht redundant ist (vgl. Kapitel \ref{chapter:metrics_semantic}). Nichtsdestotrotz gibt der \doceval auch Warnmeldungen aus, wenn er bestimmte Komponenten schlechter bewerten muss.

Aus diesen Gründen wird die Evaluation nicht mit den Metriken an sich, sondern mit den Warnmeldungen der jeweiligen Tools durchgeführt. Jedes Programm wird mit einem Verweis auf das zu analysierende Projekt aufgerufen und die Ausgaben der Programme werden in separaten Dateien umgeleitet.  Diese Dateien dienen dann als Grundlage für die spätere Evaluation. 

\subsubsection{Durchführung der Evaluation}
Wie im vorherigen Absatz beschrieben, dienen die Logdateien der drei Programme als Datengrundlage für die Evaluation. Jede Zeile in diesen Logdateien enthält mindestens den Dateipfad des gefundenen Fehlers, die Zeilennummer des Fehlers und einen Fehlercode als Zeichenkette.  Wenn die drei Programme einen übereinstimmenden Fehler finden, sollte es in allen drei Logdateien einen Eintrag geben, bei dem der Dateipfad identisch ist, die Zeilennummer innerhalb eines gewissen Intervalls identisch ist und die der Fehlercode identisch ist. Die Zeilennummer muss nicht zwingen identisch sein, da  \checkpmd die exakte Zeile des Fehlers ausgeben, sodass beispielsweise bei einem unzulässigen Wort die exakte Zeile des Wortes genannt wird, während der \doceval nur die Zeile der betroffenen Komponente ausgibt. Da die drei Tools unterschiedliche Fehlercodes ausgeben, werden diese so kategorisiert, dass Verstöße, welche von mehr als einem Tool erkannt werden, einen eigenen Fehlercode erhalten. Alle anderen Verstöße erhalten einen allgemeinen programmspezifischen Fehlercode, der somit keine näheren Informationen über die Art des Fehlers hergibt.

Basierend auf diesen Vorbereitungen kann nun für jeden Fehler jedes Programm gefunden werden, das diesen Fehler gefunden kann. Beispielsweise kann ein Fehler von allen drei Programmen, nur von \textit{Checkstyle} oder nur von \textit{PMD} und \doceval gefunden werden. Mathematisch gesehen kann die Potenzmenge der Menge \textit{\{Checkstyle, PMD, DocEvaluator\}} genommen werden, wodurch alle möglichen Kombinationen an Programmen entstehen, die ein bestimmten Fehler erkennen können. Durch Zählen der Fehler pro Programmkombination lässt sich bewerten, welche Programme besonders viele Fehler finden, die andere Programme nicht erkennen und ob alle drei Programme viele identische Fehler finden. 


\subsubsection{Auswahl der Regeln}
Zum Vergleich der drei Programme müssen zunächst Regeln festgelegt werden, bei denen die Programme eine Warnmeldung ausgeben. Bei \checkpmd{} erfolgt die Konfiguration über  Extensible Markup Language-Dateien. Beim \doceval erfolgt die Konfiguration über das in Kapitel \ref{chapter:conf} beschriebene \ac{JSON}-Format. Bei der Auswahl der Regeln muss beachtet werden, dass \checkpmd auch andere Fehler wie z.~B. komplexe Methoden finden können. Diese sind in diesem Kontext nicht relevant und werden ignoriert. Außerdem können die drei Programme zum Teil unterschiedliche Fehler finden, da beispielsweise PMD (anders als Checkstyle) den Inhalt der Dokumentation auf das Auftauchen bestimmter Begriffe prüfen kann, Checkstyle aber dafür (anders als PMD) prüfen kann, ob die Block-Tags in der richtigen Reihenfolge definiert werden. Die letztere Regel wird vom \doceval in der ausgelieferten Fassung nicht geprüft. Alle drei Programme können jedoch prüfen, ob eine Komponente dokumentiert ist oder nicht. Tabelle \ref{tab:inters_rules} vergleicht die Überschneidungen der Regeln, welche die drei Programme anwenden können. Dabei stehen die Abkürzungen \textit{CS} für \textit{Checkstyle} und \textit{DE} für \textit{DocEvaluator}:

\begin{table}[]
    \centering
    \begin{tabular}{m{4.5cm}|m{4.5cm}|m{4.5cm}}
     \textbf{CS} $\cap$ \textbf{DE}  & \textbf{PMD} $\cap$ \textbf{DE} & \textbf{PMD} $\cap$ \textbf{DE} $\cap$  \textbf{CS}  \\\hline
     \begin{itemize}
        \item Komponente dokumentiert
        \item Methode vollständig dokumentiert
         \item Fehler in Javadoc
     \end{itemize}
      & 
      \begin{itemize}
          \item Komponente dokumentiert
          \item Bestimmte Wörter in Kommentar verbieten
      \end{itemize}
      & 
       \begin{itemize}
          \item Komponente dokumentiert
         
      \end{itemize}
      \\\hline
    \end{tabular}
    \caption{Überschneidungen der Regeln der drei Programme}
    \label{tab:inters_rules}
\end{table}

Aus der Tabelle lässt sich entnehmen, dass der \doceval mit \textit{Checkstyle} die meisten Übereinstimmungen hat. Beide Programme können die Vollständigkeit der Methodendokumentation und typische Fehler in Javadoc-Kommentaren (wie z.~B. fehlerhaftes HTML) finden. PMD und der \doceval können bestimmte Wörter in der Dokumentation bemängeln, die in einer Dokumentation vermieden werden sollten. Alle drei Programme können das Vorhandensein der Dokumentation überprüfen; dabei kann auch differenziert werden, ob private Komponenten dokumentiert werden sollen oder nicht. Vor allem bei Checkstyle gibt es viele Regeln, die von den anderen beiden Programmen nicht gefunden werden und auf der Webseite \cite{checkstyle_doc_metrics} aufgelistet werden.

Zur Bewertung des \textit{DocEvaluators} sollte daher sowohl die Fehler, welche von allen Tools gefunden werden, als auch die Fehler, die nur von einzelnen Tools gefunden werden, analysiert werden. So kann  geprüft werden, wie viele Fehler von den anderen Tools gefunden werden, die der \doceval aber ignoriert, und wie viele Fehler nur vom \doceval gefunden werden. Die Ergebnisse können dann als Rückmeldung genutzt werden, um  eventuell in einer späteren Version weitere Metriken bzw. Fehler zu implementieren, die in den Open-Source-Projekten besonders häufig auftreten, aber vom \doceval aktuell nicht entdeckt werden. Interessant ist aber auch Vergleich der Ergebnisse, wenn die Regelauswahl so beschränkt wird, dass jedes Programm jeden Fehler finden sollte.

\subsection{Wahl der zu analysierenden Projekte}
Zur Durchführung der Evaluation wurden verschiedene Softwareprojekte aus GitHub heruntergeladen. Grundsätzlich kann der Vergleich mit jedem Java-Projekt durchgeführt werden, dennoch wurden einige Bedingungen festgelegt, die bei der Auswahl der Projekte eine wichtige Rolle spielen. Diese Bedingungen werden in der folgenden Auflistung präsentiert:

\begin{enumerate}
    \item Es werden nur Java-Projekte oder Projekte mit einen überragenden Anteil an Java-Code (mindestens 90 \%)) betrachtet
    \item Mindestens eine Bibliothek und ein \enquote{Standalone-Programm} (z.~dt. eigenständiges Programm) wird betrachtet
    \item  Die Projekte sollen möglichst wenige Parsing-Fehler durch ANTRL4 beim \doceval produzieren. Bei mehr als als zwei Fehlern wird dieses Projekt nicht betrachtet
    
\end{enumerate}

\subsection{Ergebnisse}


\begin{table}[]
    \centering
    \begin{tabular}{c|c|c|c}
    Name & ArgoUml & X & X\\\hline
    |DE| &1467 & &\\\hline
    |CS| &855 & &\\\hline
    |PMD|&1387 & &\\\hline
    
    \{DE\} &189 & &\\\hline
    \{CS\} &20 & &\\\hline
    \{PMD\} &113 & &\\\hline
    \{PMD, DE\} &444 & &\\\hline
    \{CS, DE\} &5 & &\\\hline
    \{PMD, CS\} & 1& &\\\hline
    \{PMD, CS, DE\} &829 & &\\\hline
    \end{tabular}
    \caption{Ergebnisse der Evaluation}
    \label{tab:eval_results}
\end{table}
\begin{comment}
\section{Unit-Test}
Dir grundlegende Vorgehensweise zur Prüfung, ob ein Programm seinen Anforderung genügt, sind Unit-Tests. Bei Unit-Tests oder auch Modul- oder Komponententests wird eine genau abgegrenzte Komponente des Programms getestet. Dabei werden die Komponente zunächst in einem bestimmten Zustand gebracht, indem beispielsweise bestimmte Eingabeparameter vorgegeben werden. Anschließend werden bestimmte Operationen auf der Komponente angewendet. Am Schluss wird der Zustand der Komponente angefragt und geprüft, ob der Zustand der Komponente nach allen Operationen so ist, wie es gemäß den Anforderungen zu erwarten ist. Gibt es Abweichung beim Zustand, so wird der Entwickler sofort informiert. 

Im Kontext des Programmms wurden die drei Arbeitspakete Traversierung, Parsing und Bewertung durch Metriken getestet.
\end{comment}