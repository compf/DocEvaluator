\label{sec:background}
In diesen Kapitel soll die Problemstellung der mangelhaften Softwaredokumentation analysiert werden. Hierzu muss zunächst der Begriff \enquote{Softwaredokumentation} definiert werden. Anschließend werden einige Statistiken präsentiert, die zeigen, dass eine mangelhafte Softwaredokumentation ein reales Problem für Entwickler. Des Weiteren sollen die Folgen von schlecht dokumentierten Code analysiert werden. Zudem wird in diesem Kapitel die Grundlagen von Javadocs erläutert, welches das Tool später als Grundlage für die Analyse der Dokumentation nutzt. Zuletzt werden noch einige Programme vorgestellt, die unter anderem auch die Dokumentation von Software bewerten können. 

\section{Definition der Softwaredokumentation}
Um den Begriff \enquote{Softwaredokumentation} zu definieren, sollte zunächst der Begriff \enquote{Dokumentation} definiert werden. Das IEEE  definiert diesen Begriff als jede textliche oder bildliche Information, welche Aktivitäten, Anforderungen, Abläufe oder Ergebnisse beschreibt, definiert, spezifiziert, berichtet oder zertifiziert \cite[S. 28]{IEEEStandardGlossaryofSoftwareEngineeringTerminology}. Kurz zusammengefasst beschreibt eine Dokumentation also wie sich eine Komponente aufgebaut ist oder wie sie sich verhält. 

Diese abstrakte Definition lässt sich so auf Softwareentwicklung übertragen. In \cite[S. 125]{Softwaredocumentationandstandards} wird Softwaredokumentation als eine Sammlung von technischen Informationen beschrieben, die für Menschen lesbar sind und die die Funktionen, Benutzung oder das Design eines Softwaresystems beschreiben . Dazu gehören beispielsweise Quellcodekommentare, UML-Diagramme oder auch Handbücher.

\section{Statistiken zu Softwarequalität}
Dass in vielen Fällen die Softwaredokumentation vernachlässigt wird, ist durch viele Studien belegt. Eine Umfrage aus dem Jahr 2002 mit 48 Teilnehmern belegt beispielsweise, dass Anforderungs- oder Spezifikationsdokumente nur selten bei Änderungen am Quellcode angepasst werden. Außerdem verwenden viele Entwickler häufig Programme, die nicht für Dokumentationszwecke entwickelt wurden (z. B Microsoft Word etc.) \cite[S. 28-29]{TheRelevanceofSoftwareDocumentationToolsandTechnologies:ASurvey}. Diese Programme sind leicht zu bedienen und sehr flexibel, verwenden jedoch teilweise proprietäre Dateiformate und sind daher nicht unbedingt effizient.

Eine weitere Studie aus dem Jahr 2019 verdeutlicht viele Aspekte aus der vorgenannten Umfrage. Es wurden dabei Daten aus Stack Overflow, Github Issues und Pull Requests und Mailing-Lists automatisiert heruntergeladen und dann von den Autoren analysiert, ob und inwieweit mit mangelhafter Softwaredokumentation zu tun haben.  Die Studie liefert klare Indizien dafür, dass die Softwaredokumentation in vielen Fällen nicht komplett, nicht auf dem neuesten Stand oder sogar nicht korrekt ist. Des Weiteren ist die Softwaredokumentation nicht gut nutzbar oder schlecht lesbar, sodass der Vorteil verloren geht\cite[S.1201 -1204]{SoftwareDocumentationIssuesUnveiled}. 
\section{Javadoc}
Javadoc \cite{Javadoc} ist ein Tool zur Generierung von Dokumentationen, das sich als de-facto Standard für Dokumentationen in der Programmiersprache Java etabliert hat \cite[S. 249]{JavadocViolationsandTheirEvolutioninOpen-SourceSoftware}.  Javadoc besteht aus speziellen Java-Kommentaren, die an bestimmten Stellen im Quellcode eingefügt werden und bei der Kompilation nicht berücksichtigt werden. Ein Javadoc-Block beschreibt immer eine bestimmted Modul (z. B. eine Klasse, Methode oder Feld).Es beginnt mit der Zeichenkette \enquote{/**}, wobei die ersten beiden Zeichen \enquote{/*} den Beginn eines mehrzeiligen Kommentars in Java einläuten, und endet mit \enquote{*/}. Zunächst sollte am Anfang des Blocks eine generelle Zusammenfassung der Komponente geschrieben werden. Danach können sogenannte Tags, die mit dem \enquote{@}-Zeichen beginnen, benutzt werden. diese beschreiben wiederum einen bestimmten Teilbereich einer Komponente. Tabelle \ref{tab:table_javadoc_method} beschreibt einige Tags für Java-Methoden.
\begin{table}[h]
    \centering
    \begin{tabular}{m{4cm}|m{4cm}|m{7cm}}
    Tag & zusätzliche Parameter &Beschreibung\\
    \hline
        @param  & Paramatername & Beschreibt einen Methodenparameter\\
        \hline
         @return & & Beschreibt den Rückgabewert der Methode, sofern er existiert \\
         \hline
         @throws &Exception & Beschreibt welche Exceptions diese Methode werfen kann und möglichst unter welchen Umständen dies passiert \\
           \hline
         @since & Versionsnummer & Die Versionsnummer der Software oder des Packets, in dem diese Methode zuerst implementiert wurde\\
           \hline
         @deprecrated & & Falls diese Methode veraltet ist und nicht mehr verwendet werden sollte, kann hier eine Alternative beschrieben werden. \\
           \hline
         
           \hline
         
         
         
         
    \end{tabular}
    \caption{Wichtige Javadoc-Tags}
    \label{tab:table_javadoc_method}
\end{table}