\label{sec:background}
In diesem Kapitel soll die Problemstellung der mangelhaften Softwaredokumentation analysiert werden. Hierzu muss zunächst der Begriff \enquote{Softwaredokumentation} definiert werden. Anschließend werden einige Statistiken präsentiert, die zeigen, dass eine mangelhafte Softwaredokumentation ein reales Problem für Entwickler ist. Des Weiteren sollen die Folgen von schlecht dokumentierten Code analysiert werden. Zudem wird in diesem Kapitel die Grundlagen von Javadocs erläutert, welches das Tool später als Grundlage für die Analyse der Dokumentation nutzt. Zuletzt werden noch einige Programme vorgestellt, die unter anderem auch die Dokumentation von Software bewerten können. 

\section{Softwaredokumentation}
Um den Begriff \enquote{Softwaredokumentation} zu definieren, sollte zunächst der Begriff \enquote{Dokumentation} definiert werden. Das IEEE  definiert diesen Begriff als jede textliche oder bildliche Information, welche Aktivitäten, Anforderungen, Abläufe oder Ergebnisse beschreibt, definiert, spezifiziert, berichtet oder zertifiziert \cite[S. 28]{IEEEStandardGlossaryofSoftwareEngineeringTerminology}. Somit beschreibt eine Dokumentation, wie sich eine Komponente aufgebaut ist oder wie sie sich verhält.

Diese abstrakte Definition lässt sich so auf Softwareentwicklung übertragen. In \cite[S. 125]{Softwaredocumentationandstandards} wird Softwaredokumentation als eine Sammlung von technischen Informationen dargestellt, die für Menschen lesbar sind und welche die Funktionen, Benutzung oder das Design eines Softwaresystems beschreiben. So beschreibt Donald E. Knuth in \cite[S. 97]{LiterateProgramming}, dass die Hauptaufgabe beim Programmieren nicht sein sollte, einen Computer zu erklären, was er machen sollte, sondern anderen Menschen zu erklären, was der Computer machen sollte.

Im Kontext dieser Bachelorarbeit sollen allerdings nur bestimmte Arten der Softwaredokumentation betrachtet werden, da eine umfassende Betrachtung innerhalb der vorgegebenen Zeit nicht möglich ist. 
Für diese Bachelorarbeit werden daher nur bestimmte Inline-Kommentare im Quellcode betrachtet, die ein Spezialfall von normalen mehrzeiligen Kommentaren sind. Diese Kommentare werden wie normale Kommentare erkannt und werden daher nicht Bestandteil des kompilierten Programms. Nichtsdestotrotz haben diese spezifischen Kommentare aber eine bestimmte Struktur, die eine leichte Verarbeitung durch Computerprogramme ermöglicht und gleichzeitig trotzdem für Menschen lesbar bleibt. In vielen Fällen werden diese Kommentare einer bestimmten Komponente zugeordnet, wobei dies oft dadurch geschieht, dass der Kommentar direkt vor dieser Komponente steht. Ein Beispiel dafür ist Javadoc, da jeder Javadoc-Kommentar auch ein gültiger mehrzeiliger Kommentar ist. 

Andere Möglichkeiten zur Dokumentation, die nur hier kurz erwähnt werden sollen, sind UML-Diagramme, Handbücher und  Readme-Dateien.



\section{Javadoc}
Javadoc \footnote{\href{https://www.oracle.com/java/technologies/javase/javadoc-tool.html}{Javadoc Website (besucht am 07.01.2022)} } ist ein Tool zur Generierung von Dokumentationen, das sich als de-facto Standard für Dokumentationen in der Programmiersprache Java etabliert hat \cite[S. 249]{JavadocViolationsandTheirEvolutioninOpen-SourceSoftware}.  Javadoc besteht aus speziellen Java-Kommentaren, die an bestimmten Stellen im Quellcode eingefügt werden und daher bei der Kompilation nicht berücksichtigt werden. Ein Javadoc-Block beschreibt immer ein bestimmtes Modul (z. B. eine Klasse, Methode oder Feld).Es beginnt mit der Zeichenkette \enquote{/**}, wobei die ersten beiden Zeichen \enquote{/*} den Beginn eines mehrzeiligen Kommentars in Java einläuten, und endet mit \enquote{*/}. Durch das zusätzliche \enquote{*} unterscheidet sich ein Javadoc-Kommentar von einem normalen mehrzeiligen Kommentar, der zwar zur Kommentierung eines Blocks verwendet werden kann, aber vom Javadoc-Tool ignoriert wird und daher einen geringeren Mehrwert hat.
		\begin{figure}[ht!]
			\lstinputlisting
			[caption={Beispielhafter Javdoc-Block für einfache Methode},
			label={lst:simple_javadoc},
			captionpos=b,language=java, basicstyle=\footnotesize, tabsize=1, showstringspaces=false,  numbers=left]
			{figures/ternary.java}
		\end{figure}
Zunächst sollte am Anfang des Blocks eine generelle Zusammenfassung der Komponente geschrieben werden. Danach können sogenannte Tags (z.~dt. Auszeichner), die mit dem \enquote{@}-Zeichen beginnen, benutzt werden. Diese beschreiben wiederum einen bestimmten Teilbereich einer Komponente. Es ist zudem Konvention, dass jede Zeile in einem Javdoc-Block mit einem Asterisk beginnt. 

Quelltext \ref{lst:simple_javadoc} zeigt ein Beispiel für eine gelungene Verwendung von Javadoc. Zunächst wird der Zweck der Methode beschrieben, anschließend wird jeder Parameter erläutert. Dabei sollte in komplexeren Fällen auch erklärt werden, welche Werte gültig für den Parameter sind. Danach folgt eine Beschreibung des Rückgabewertes, welche am besten auch jeden möglichen Fall abdeckt. Mit \enquote{{@code ...}} kann auf einen Parameter referenziert werden. Mit diesen Informationen kann der Programmierer leicht überblicken, wie eine Methode genutzt werden, sodass die Einarbeitungszeit und die Fehleranfälligkeit reduziert werden kann.  



Tabelle \ref{tab:table_javadoc_method} beschreibt einige Tags für Java-Methoden:
\begin{table}[h]
    \centering
    \begin{tabular}{m{4cm}|m{4cm}|m{7cm}}
    Tag & zusätzliche Parameter &Beschreibung\\
    \hline
        @param  & Paramatername & Beschreibt einen Methodenparameter\\
        \hline
         @return & & Beschreibt den Rückgabewert der Methode, sofern er existiert \\
         \hline
         @throws &Exception & Beschreibt welche Exceptions diese Methode werfen kann und möglichst unter welchen Umständen dies passiert \\
           \hline
         @deprecrated & & Falls diese Methode veraltet ist und nicht mehr verwendet werden sollte, kann hier eine Alternative beschrieben werden. \\
           \hline
         
           \hline
         
         
         
         
    \end{tabular}
    \caption{Wichtige Javadoc-Tags}
    \label{tab:table_javadoc_method}
\end{table}




Nachfolgend ist ein Auszug von empfehlenswerten Tipps von der Oracle-Webseite \footnote{\href{https://www.oracle.com/technical-resources/articles/java/javadoc-tool.html}{Javadoc Style-Guide (besucht am 07.01.2022)} }:
\begin{itemize}
    \item Nicht in jeden Fall vollständige Sätze verwenden, aber klar formulieren, was die Aufgabe einer Komponente ist
    \item In der dritten und nicht in der zweiten Person schreiben
    \item Nicht repetitiv sein. Ein Kommentar, der im Wesentlichen nur den Namen einer Komponente wiedergibt, hat keinen Mehrwert
    \item  Beschreibungen von Methoden sollten mit einem Verb beginnen
    \item bei einem Verweis auf das aktuelle Objekt sollte das spezifische \textit{this} statt des allgemeineren \textit{the} verwendet werden, bspw. \enquote{opens the connection of \textbf{this} object} statt \enquote{opens the connection of \textbf{the} object}
    \item Bezeichner sollten in mit \textit{<code></code>} umschlossen werden, um deutlich zu machen, dass dies eine andere Komponente ist
    \item Der Kommentar sollte eventuelle Unterschiede unter verschiedene Plattformen erläutern
    \item Die Dokumentation sollte erläutern, wie sich die Komponente in Randfällen verhält
    
\end{itemize}
Diese Javadoc-Blöcke können dann von dem gleichnamigen Tool in eine HTML-Datei umgewandelt werden und ermöglichen den Entwicklern damit einen komfortablen Überblick über alle Komponenten eines Moduls. Zudem könne Javadoc-Blöcke ebenfalls HTML-Inhalte besitzen, die dann von Javadoc in die HTML-Datei übernommen werden, sodass der Entwickler beispielsweise Tabellen zur übersichtlichen Präsentation  von Informationen verwenden kann. Abbildung \ref{fig:javadoc_example_screenshot} zeigt, wie eine Methode mittels Javadoc in gerendertes HTML beschrieben wird. 
\begin{figure}[h]
    \centering
    \includegraphics[width=\columnwidth]{figures/javadoc_screenshot.png}
    \caption{Gerenderte HTML-Ausgabe von Javadoc}
    \label{fig:javadoc_example_screenshot}
\end{figure}

Eine \ac{IDE} kann die Informationen auslesen und dann beispielsweise bei der Autovervollständigung nutzen, um den Entwickler beim Schreiben des Programmcodes unterstützen. So kann sich ein Programmierer auch in einer fremden \ac{API} leichter zurechtfinden.

Ein Javadoc-Kommentar wird vererbt und muss daher für eine abgeleitete Klasse nicht neu geschrieben oder redundant geklont werden. Dies ist sinnvoll, da abgeleitete Klassen einen Vertrag erfüllen müssen, der bei einer guten Dokumentation auch schon in der Javadoc-Dokumentation beschrieben wird. Auch bei Methoden, die aufgrund einer Schnittstelle implementiert werden müssen, ist eine Neudefinition des Javadoc-Kommentar unnötig. Falls sinnvoll, kann aber dennoch ein eigener Javadoc-Kommentar erstellt werden, der allerdings den Kommentar der Quelle vollständig ersetzt. Mit \textit{@inheritDoc} kann der ursprüngliche Kommentar aber trotzdem eingefügt werden.

Für andere Programmiersprachen gibt es vergleichbare Werkzeuge, die ähnliche Funktionen anbieten und bei denen die Dokumentationen mit einer relativ ähnlichen Syntax erstellt werden. Dazu gehören beispielsweise Doxygen für C++-Programme oder der PHP-Documentor für PHP-Programme. 



\section{GitHub Actions}\label{chapter:github_actions}

Github Action \footnote{\href{https://github.com/features/actions}{GitHub Actions Website (besucht am 07.01.2022 }} ist eine von GitHub angebotene Plattform zur Vereinfachung des \ac{CI/CD}'s. Mithilfe von  Github Actions wird Programmcode ausgeführt, wenn ein bestimmtes Ereignis stattfindet. Dieses Ereignis kann z. B. ein Push-Ereignis sein, bei dem neuer Quellcode in das GitHub-Repository hochgeladen wird, oder eine neue Version des Programms zum Release freigegeben wird. Der Programmcode kann wahlweise auf einem von  GitHub vorbereiteten System ausgeführt oder auf einem eigenes System ausgeführt werden. Schlägt der Programmcode fehl kann der Nutzer den Grund des Fehlers über die  \enquote{Actions}-Registerkarte herausfinden.

Dort kann der Nutzer auch sämtliche Ausgaben des Programms ansehen, die auf der Konsole ausgegeben werden. Die folgende nummerierte Liste beschreibt grob, wie Programmcode in GitHub Actions ausgeführt werden kann:
\begin{enumerate}
    \item Ein Ereignis tritt ein, indem beispielsweise neuer Programmcode mittels Push hochgeladen wird 
    \item Github wählt ein geeignetes System aus (z. B. eine virtuelle Maschine oder ein vom Programmierer hierfür konfiguriertes System
    \item Auf dem System wird der aktuelle Programmcode des Repositorys geklont (der Branch kann frei bestimmt werden) 
    \item Der Programmcode ausgeführt, dabei steht der Pfad zum geklonten Repository über die Umgebungsvariable \textit{\$GITHUB\_WORKSPACE} bereit
    \item Abhängig vom Erfolg der Ausführung des Programmcodes:
    \begin{enumerate}
        \item Bei einem Erfolg wird der Programmierer mittels eines grünen Häkchens informiert
        \item Bei einer Fehlermeldung wird der Programmierer mittels eines roten Kreuzes über den Fehlschlag informiert und hat auch Zugriff auf die Fehlermeldungen. Je nach Einstellung des Repositorys können bestimmte Aktivitäten dann auch gestoppt werden, damit fehlerhafter Programmcode nicht weiter verbreitet wird. 
    \end{enumerate}
\end{enumerate}

Ein Anwendungsfall von GitHub Actions sind automatisierte Tests. Bei einem Push-Ereignis kann der aktuelle Programmcode mit einer geeigneten Testbibliothek getestet werden, sodass im Falle eines fehlgeschlagenen Tests der Programmierer informiert wird und die notwendigen Änderungen veranlassen kann: 

\subsection{Wichtige Begriffe im Zusammenhang mit GitHub Action}
In der folgenden Auflistung werden einige wichtige Begriffe erläutert, die im Zusammenhang mit GitHub Actions verwendet werden und deren Kenntnis zum Verständnis wichtig ist. 
\begin{description}
    \item[Workflow] Ein Workflow (Ablauf) ist ein automatisierter Prozess, der verschiedene Jobs ausführt, wenn ein Ereignis eintritt. Der Worklflow wird durch eine YAML-Datei definiert
    \item[Job] Ein Job ist eine Ansammlung von sogenannten \enquote{Steps}, die man als Befehle interpretieren kann. Alle Befehle in einem Job werden auf der gleichen Maschine ausgeführt. Alle Jobs in einem Workflow werden standardmäßig parallel ausgeführt, da keine Abhängigkeit zwischen den Jobs vermutet wird.
    \item[Step] Ein Step (Schritt) ist ein Shell-Befehl oder ein Verweis auf eine andere GitHub Action, die von einem Job ausgeführt werden soll. Innerhalb eines Jobs werden die Steps in der definierten Reihenfolge ausgeführt. 
    \item[Action] Eine Action ist ein Programm, das in Workflows eingebunden werden kann und häufig vorkommende Aufgaben erledigt. Dies hat den Vorteil, dass Redundanz vermieden wird. Im offiziellen Marketplace von GitHub lassen sich zu vielen verschiedenen Aufgaben bereits Actions finden. Es können aber auch eigene Actions definiert werden, was weiter unten beschrieben wird. 
    \item[Runner] Als Runner wird das System bezeichnet, auf dem die Jobs ausgeführt werden. Dies ist standardmäßig ein Linux-Betriebssystem; es können aber auch Windows oder macOS zur Ausführung des Codes verwendet werden. Bei besonderen Anforderungen kann der Runner auch auf einem System laufen, auf dem der Programmierer selbst Zugriff hat.
\end{description}
\subsection{Erstellung eines Workflows}
Ein Workflow kann über die Registerkarte \enquote{Actions} erstellt werden und wird intern in dem Verzeichnis \textit{.github/workflows} gespeichert. Abbildung \ref{lst:simple_workflow} illustriert eine typische Workflow-Datei, die standardmäßig erstellt wird. Der Code wird in Tabelle \ref{tab:descr_example_workflow} genauer erklärt.
\clearpage
\begin{table}[h!]
    \centering
    \begin{tabular}{c|m{10cm}}
        Zeile(n) &  Beschreibung \\\hline
         1 & Der Name des Workflows\\\hline
         2-6 & Hier werden die verschiedenen Ereignisse beschrieben, die den Workflow auslösen. Außerdem werden die Branches spezifiert, die auf das Ereignis überwacht werden sollenö\\\hline
         7 &  Dies erlaubt, dass ein Workflow auch manuell angestoßen wird, etwa zu Debugging-Zwecken\\\hline
         8-9 & Erstellt einen Job mit dem Namen \enquote{Build}\\\hline
         10 & Bestimmt auf welchen System der Workflow läuft\\\hline
         11 & Beginn der einzelnen Steps\\\hline
         12 & Bestimmt, dass der Quellcode des Repositorys geklont wird und dessen Pfad über eine Umgebungsvariable zugängig gemacht wird\\\hline
         13-14 & Gibt hier als Beispiel \enquote{Hello World} aus\\\hline
         16-19 &Beispiel um mehrere Befehle sequentiell auszuführen\\\hline
         
    \end{tabular}
    \caption{Beispielhafter Workflow}
    \label{tab:descr_example_workflow}
\end{table}
		\begin{figure}[h!]
			\lstinputlisting
			[caption={Beispielhafter Workflow-Datei},
			label={lst:simple_workflow},
			captionpos=b, basicstyle=\footnotesize, tabsize=1, showstringspaces=false,  numbers=left]
			{figures/workflow_example.yaml}
		\end{figure}
	\clearpage
	\subsection{Ausführung von Workflows}
	Wenn ein Workflow durch ein Ereignis ausgeführt wird, lässt sich die Ausgabe des Workflows über die Registerkarte \enquote{Actions} anzeigen. Dabei erhält der Programmierer auch Informationen darüber, ob der Workflow erfolgreich war oder fehlgeschlagen ist. Abbildung \ref{fig:workflow_overview} zeigt die Übersicht der ausgeführten Workflows. Wie der Abbildung zu entnehmen ist, sieht der Programmierer den Commit, auf dem der Workflow ausgeführt wurde. Auf der rechten Seite befindet sich auch eine Angabe, wann dieser Workflow ausgeführt wurde. Zudem enthält die Übersicht die Information, wie lange der Prozess gedauert hat. 
	\begin{figure}[h]
	    \centering
	    
	    \includegraphics[width=\columnwidth]{figures/workflow_overview.png}
	    \caption{Übersicht über ausgeführten Workflows in GitHub Actions}
	    \label{fig:workflow_overview}
	\end{figure}
	
	Durch einen Klick auf den Titel des Commits, wechselt Github auf eine weitere Seite, die grundsätzlich die gleichen Informationen wie zuvor enthält. Es ist jedoch hier möglich direkt zu der YAML-Datei zu gelangen, die diesen Workflow beschreibt. Ein weiterer Klick auf den Button mit dem Namen von dem definierten Job wechselt auf eine weitere Seite, die die einzelnen Steps des Jobs auflistet und es ermöglicht, die Konsolenausgabe jedes Steps anzuzeigen. Auch eine Aufsplittung der Zeitdauer der einzelnen Steps wird angezeigt, sodass der Programmierer weiß, wie viel Zeit jeder Step benötigt. Abbildung \ref{fig:workflow_output} zeigt, wie eine solche Ausgabe auf der Konsole aussieht:
	\begin{figure}[h]
	    \centering
	    
	    \includegraphics[width=\columnwidth]{figures/workflow_output.png}
	    \caption{Übersicht über die ausgeführten Workflows}
	    \label{fig:workflow_output}
	\end{figure}
\subsection{Erstellung einer eigenen Action}
Um eine eigene Github Action zu erstellen, muss in dem Hauptverzeichnis des Repositorys, in dem der Programmcode der Action liegt, eine Datei namens \enquote{action.yaml} oder \enquote{action.yml} erstellt werden. Abbildung \ref{lst:create_action_example} zeigt eine beispielshafte \enquote{action.yaml}
	\begin{figure}[h!]
			\lstinputlisting
			[caption={Beispielhafte Action-Konfigurationsdatei},
			label={lst:create_action_example},
			captionpos=b, basicstyle=\footnotesize, tabsize=1, showstringspaces=false,  numbers=left]
			{figures/create_action_example.yaml}
		\end{figure}
Zunächst wird der Name der Action und eine Beschreibung definiert. Anschließend können Eingabeparameter definiert werden, die später im Programm verwendet werden können. Zu jedem Parameter kann auch festgelegt werden, ob er zwingend erforderlich ist und den Standardwert des Parameters. Außerdem können Ausgabeparameter festgelegt werden, die spätere Actions als Eingabe nutzen können. Danach wird festgelegt, wie diese Action ausgeführt wird. Eine Action kann in einer JavaScript-Umgebung, in einem Docker-Container, oder als Liste von verschiedenen Steps ausgeführt werden. Dies wird in den folgenden Unterabschnitten kurz beschrieben.
\subsubsection{Docker-Umgebung}
In einer Docker-Umgebung kann die Action ausgeführt werden, ohne dass der Nutzer der Action sich darüber Gedanken machen muss, welche Abhängigkeiten oder Voraussetzungen diese Action benötigt. Dies ermöglicht eine fehlertolerante Ausführung des Programmcodes. Das System kann dabei sehr frei konfiguriert werden. Allerdings laufen Actions in einer Docker-Umgebung langsamer als in einer JavaScript-Umgebung. Außerdem kann zurzeit ein Action in einer Docker-Umgebung nur mittels eines Runners in Linux ausgeführt werden.

\subsubsection{JavaScript-Umgebung}
Da viele GitHub Actions in JavaScript programmiert werden, ist diese Umgebung als Option vorhanden. Es kann direkt eine geeignete Node-Umgebung installiert werden und eine JavaScript-Datei mittels \textit{main} ausgeführt werden. 

\subsubsection{Composite-Action}
Eine sogenannte Composite-Action ermöglicht es mehrere Actions in eine Action zu vereinen und so wieder Code-Redundanz zu vermeiden. Die Syntax ist dabei ähnlich wie ein normaler Workflow. 
