\chapter{Umsetzung}
In diesem Kapitel wird auf die Umsetzung der in Kapitel \ref{chapter_conception} beschriebene Architektur eingegangen. Dazu werden Informationen über die Einrichtung und Konfiguration des Tools gegeben.
\section{Implementierung von ANTLR4}\label{chapter:antlr4_impl}
Für die Programmiersprache Java steht eine Grammatik, die auf Github unter der BSD-Lizenz angeboten wird ist, zur Verfügung\footnote{\href{https://github.com/antlr/grammars-v4/tree/master/java/java}{Grammatik-Dateien für Java (besucht 07.01.2022)}}, allerdings ignoriert diese Grammatik alle Kommentare. Daher müssen einige Änderungen sowohl am Lexer als auch am Parser vorgenommen werden. Im Lexer werden standardmäßig alle Tokens in einem Kommentar in einen versteckten Kanal gespeichert, was dazu führt, dass diese Tokens vom Parser ignoriert werden. Daher wird dieses Verhalten durch Definition eines neuen Tokens so geändert, dass Javadoc-Kommentare auch vom Parser verarbeitet werden können, aber mehrzeilige und einzeilige Kommentare weiterhin ignoriert werden. Einzeilige Kommentare sind hier nicht relevant, da sie kein Javadoc enthalten.

Mehrzeilige Kommentare könnten theoretisch auch berücksichtigt werden, da einige Entwickler diese anstelle von Javadoc benutzen. Allerdings werden solche mehrzeilige Kommentare vor Komponenten nicht von Tools erkannt und haben daher einen geringeren, aber durchaus vorhandenen Nutzen \cite[S. 4]{HowDocumentationEvolvesoverTime}. Deshalb werden Komponenten, die zwar mit mehrzeiligen Kommentaren aber nicht mit Javadoc dokumentiert sind, wie undokumentierte Komponenten betrachtet. Für einen Entwickler sollte es so schnell möglich sein, solche nicht korrekt dokumentierten Komponenten zu identifizieren und deren mehrzeilige Kommentare in gültige Javadoc-Kommentare umzuwandeln und so die Qualität der Dokumentation zu erhöhen. Für andere Programmiersprachen können jedoch normale mehrzeilige wie strukturierte Kommentare betrachtet werden, wenn dies für sinnvollerer erachtet wird.

Mehr Änderungen müssen an der entsprechenden Parser-Datei \textit{JavaParser.g4} durchgeführt werden.  Da diese Änderungen für die eigentliche Thematik dieser Bachelorarbeit nur eine untergeordnete Rolle spielen, wird hier nicht jede Änderung genauer erklärt. Tabelle \ref{tab:parser_changes} im Anhang listet alle Änderungen an der Parserdatei auf. 

\section{Konfiguration und Nutzung des Tools}
Zur Nutzung des Tools werden bestimmte Informationen benötigt, die aus verschiedenen Quellen bezogen werden. Zunächst benötigt das Tool den Pfad, der die Quelldateien enthält, die nach Kapitel \ref{chapter:traversing} traversiert werden sollen. Dieser wird als namenloser Parameter über die Kommandozeile übergeben. Er ist optional, da bei dessen Fehlen das aktuelle Arbeitsverzeichnis genommen wird. Die weiteren Information werden aus zwei Quellen bezogen. Die erste Quelle ist eine \ac{JSON}-Datei namens \enquote{comment\_conf.json},welche als notwendige Daten für die Arbeit des Programms dienen. Listing \ref{lst:example_conf} zeigt ein beispielhafte \ac{JSON}-Datei:

\begin{figure} [htbp]
\lstinputlisting
[caption={Beispielhafte Konfigurationsdatei für das Tool},
label={lst:example_conf},
captionpos=b, basicstyle=\footnotesize, tabsize=2, showstringspaces=false,  numbers=left]
{figures/example_conf.json}
\end{figure}

In dieser Beispielsdatei  werden alle Dateien mit der Dateieendung \enquote{.java} bei der Traversierung betrachtet (Z. 1). Außerdem werden dabei keine Datei bei der ausgeschlossen (Z. 2). Diese beiden Werte entsprechend dabei ihre Standardwerte. Sie könnten also bei dieser Konfigurationsdatei weggelassen werden und das Programmverhalten würde sich nicht ändern.

Anschließend (Z. 4-8) werden die zu verwendenden Metriken definiert. Jede Metrik besitzt einen \textit{metric\_name}, der den Typ der Metrik spezifiziert. In Zeile 5 wäre dies beispielhaft die Metrik \enquote{Anteil der dokumentierten Komponenten an allen Komponenten}( \ref{chapter:metrics_simple_comment}). Tabelle \ref{table:metrics_name} listet alle Namen der Metriken auf. Diese Namen werden vom Metrikmanager dazu genutzt, um die passende Klasse zu finden und so ein Metrikobjekt zu erzeugen. Außerdem erhält jede Metrik durch \textit{unique\_name} einen eindeutigen Namen. Dieser kann auch weggelassen werden; dann wird der eindeutige Name aus dem Namen der Metrik und einer fortlaufenden Nummerierung erzeugt. Zuletzt besitzt jede Metrik das Attribut \textit{weight}, welches zur Bestimmung der Relevanz bzw. des Gewichts der Metrik dient und von einem \textit{MetricResultBuilder} zur Bestimmung eines Gesamtergebnis benutzt werden kann. Ein \textit{MetricResultBuilder}, der keine Gewichtung der Metriken benötigt, wird diese Information ignorieren. Das Gewicht ist ebenfalls optional; bei dessen Fehlen wird das Gewicht \enquote{1} eingesetzt. 

Als nächstes (Z. 9) wird der Schwellwert festgelegt. Dieser Wert legt fest, ob das Programm beim Unterschreiten dieses Wertes mit einer Fehlermeldung abbrechen soll. In den darauf folgenden Zeilen 10 und 11 wird der \textit{MetricResultBuilder} für die Metriken und Dateien festgelegt. In dem Beispiel werden die Teilresultate der Metriken mittels eines gewichteten Mittelwert zu einem Gesamtergebnis aggregiert. Die Teilergebnisse der Dateien werden dagegen mittels des Standard-Algorithmus zusammengefasst, welches der arithmetische Mittelwert ist. 
\subsection{Einbindung in GitHub Actions}\label{chapter:github_actions_impl}
Um das Tool in GitHub Actions einzubinden, müssen einige Schritte erfolgen. Zunächst muss eine \enquote{action.yaml} geschrieben werden, die das GitHub-Repository als Aktion markiert und die notwendige Befehle für die Ausführung enthält. Listing  \ref{lst:action} zeigt den kompletten Code einer Aktion:
\begin{figure} [htbp]
\lstinputlisting
[caption={Beispielhafte Action-Datei für das Tool},
label={lst:action},
captionpos=b, basicstyle=\footnotesize, tabsize=2, showstringspaces=false,  numbers=left]
{figures/action.yml}
\end{figure}

In den ersten beiden Zeilen werden Attribute wie der Name und eine Beschreibung gesetzt. Danach (Z. 4-7) wird der Eingabeparameter für die minimal erlaubte Bewertung für die Dokumentationsqualität definiert, damit dieser von den Nutzern der Aktion verändert werden kann.  Außerdem werden noch ein Icon und eine Farbe gesetzt (Z. 8-10), da dies von GitHub Actions vorgeschrieben ist. In den Zeilen 11 bis 13 ist der wichtige Programmcode enthalten, in denen die Aktion als JavaScript-Aktion mit der Node-Version 16 festgelegt wird. Zudem enthält die letzte Zeile auch den Pfad zur Quellcodedatei, mit dem das Programm gestartet werden soll. 

\bigskip
Da das Programm in TypeScript programmiert wurde, eine JavaScript-Aktion aber reines JavaScript benötigt, sind weitere Schritte nötig. Es wird ein weiterer Workflow benötigt, der bei jedem Push in dem Main-Zweig folgende Schritte ausführt:
\begin{enumerate}
    \item Klonen des Main-Branch des Repositories(wie bei den meisten anderen Workflows)
    \item Aufruf von TSC, Konvertierung des TypeScript-Codes in JavaScript
    \item Aufruf und Benutzung von NCC\footnote{\href{https://github.com/vercel/ncc}{NCC GitHub-Repository (besucht 07.01.2022)}}. Packen aller JavaScript-Dateien in eine einzige Datei
    \item Kopieren der generierten Datei, die den gesamten Quellcode enthält und der \enquote{action.yml}, in eine (neue) Branch \textit{action}. Dies wird mittels der Aktion \textit{Branch-Push}\cite{Branch-Push} durchgeführt
\end{enumerate}
Durch diese Schritte wird eine neue Branch erstellt, die nur die notwendige JavaScript-Datei und die \textit{action.yml} enthält. Dadurch können Nutzer der Aktion diese schneller herunterladen und nutzen. Es wäre auch möglich, kein \enquote{NCC} zu verwenden, also alle Javascript-Dateien in die neue Branch zu kopieren, allerdings ist die hier gewählte Methode praktikabler, da dann nur ein Lesezugriff beim Starten des Programms erforderlich ist und so ein Geschwindigkeitsvorteil existiert. 

\subsubsection{Nutzung der Aktion}

Die oben erstellte Aktion kann nun von jedem GitHub-Repository verwendet werden. dazu kann der folgende Listing \ref{lst:action_using} als zusätzlicher Schritt in einem Workflow eingebunden werden. 
\begin{figure} [htbp]
\lstinputlisting
[caption={Verwendung der Aktion in einem Workflow},
label={lst:action_using},
captionpos=b, basicstyle=\footnotesize, tabsize=2, showstringspaces=false,  numbers=left]
{figures/action_using.yml}
\end{figure}

Hier wird die aktuelle Version des JavadocEvaluators aus der Branch \textit{action} heruntergeladen und automatisch ausgeführt. Als Parameter wird beispielsweise ein Grenzwert von 20 übergeben, der jedoch nach Belieben angepasst werden kann. Wenn das entsprechende Ereignis des Workflow eintritt (z. B. ein Push-Ereignis), wird der JavadocEvaluator mit dem Parameter aufgerufen und zeigt unter der Registerkarte \textit{Actions} eine Fehlermeldung an, wenn die Dokumentationsqualität den Grenzwert unterschreitet und somit nicht ausreichend ist.
