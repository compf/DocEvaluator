\chapter{Umsetzung}
\section{Implementierung von ANTLR4}\label{chapter:antlr4_impl}
Für die Programmiersprache Java steht bereits eine Grammatik, die auf Github unter der BSD-Lizenz angeboten wird ist, zur Verfügung\footnote{\href{https://github.com/antlr/grammars-v4/tree/master/java/java}{Grammatik-Dateien für Java (besucht 07.01.2022)}}, allerdings ignoriert diese Grammatik alle Kommentare. Daher mussten einige Änderungen sowohl am Lexer als auch am Parser vorgenommen werden. Im Lexer werden standardmäßig alle Tokens in einem Kommentar in einen versteckten Kanal gespeichert, was dazu führt, dass diese Tokens vom Parser ignoriert werden. Daher wurde dieses Verhalten durch Definition eines neuen Tokens so geändert, dass Javadoc-Kommentare auch vom Parser verarbeitet werden können, aber mehrzeilige und einzeilige Kommentare weiterhin ignoriert werden. Einzeilige Kommentare sind hier nicht relevant, da sie kein Javadoc enthalten. Mehrzeilige Kommentare könnten theoretisch auch berücksichtigt werden, da einige Entwickler diese anstelle von Javadoc benutzen. Allerdings werden solche mehrzeilige Kommentare vor Komponenten nicht von Tools erkannt und haben daher einen geringeren, aber durchaus vorhandenen Nutzen \cite[S. 4]{HowDocumentationEvolvesoverTime}. Deshalb werden Komponenten, die zwar mit mehrzeiligen Kommentaren aber nicht mit Javadoc dokumentiert sind, wie undokumentierte Komponenten zu betrachten. Für einen Entwickler sollte es so schnell möglich sein, solche nicht korrekt dokumentierten Komponenten zu identifizieren und deren mehrzeilige Kommentare in gültige Javadoc-Kommentare umzuwandeln und so die Qualität der Dokumentation zu erhöhen. Für andere Programmiersprachen können jedoch normale mehrzeilige wie strukturierte Kommentare betrachtet werden, wenn dies für sinnvollerer erachtet wird.

Mehr Änderungen müssen an der entsprechenden Parser-Datei \textit{JavaParser.g4} durchgeführt werden.  Tabelle \ref{tab:parser_changes} im Anhang listet alle Änderungen an der Parserdatei auf:
