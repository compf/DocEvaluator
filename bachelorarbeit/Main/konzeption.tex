\begingroup
\renewcommand{\cleardoublepage}{} % TODO maybe removing this and next
\renewcommand{\clearpage}{}
\chapter{Konzeption}\label{chapter_conception}
\endgroup
Im Folgenden wird ein Konzept für die Implementierung des Tools vorgestellt, das  die Ziele der Bachelorarbeit erfüllen soll.
\section{Traversierung aller relevanten Dateien}\label{chapter:traversing}
Softwareprojekte bestehen aus Hunderten von Dateien, die nicht alle Quellcode enthalten. Beispielsweise gehören Konfigurationsdateien, Ressourcedateien wie Bilder oder binäre Dateien zu den Dateien, bei denen eine Analyse der Softwaredokumentation im Hinblick auf die begrenzte Zeit für die Bachelorarbeit nicht implementierbar ist. Daher ist es sinnvoll, bestimmte Dateien bei der Analyse auszuschließen beziehungsweise nur bestimmte Dateien zu betrachten. Bei einer Weiterentwicklung des Tools nach Abschluss der Bachelorarbeit kann das Tool auf andere Dateitypen ausgeweitet werden, um so ein besseres Gesamtbild über die Softwaredokumentation zu erhalten.

Um die relevanten Dateien zu finden, wird zunächst ein übergeordnetes Verzeichnis benötigt, was bei Softwareprojekten aber der Standard sein sollte. Dieses Verzeichnis kann dann rekursiv durchlaufen werden und somit die Liste aller darin gespeicherten Dateien abgerufen werden. Die relevanten Dateien können dann durch Überprüfung ihres Dateinamens mittels bestimmter Regeln ermittelt werden, die der Benutzer des Tools festlegen kann.

Beim JavadocEvaluator wird hierzu die NPM-Bibliothek Minimatch \footnote{\href{https://github.com/isaacs/minimatch}{Minimatch GitHub-Repository (besucht am 07.01.2022)}} verwendet, die es ermöglicht, Dateinamen mit Wildcard-Patterns zu vergleichen. Zum Beispiel könnte der Dateiname \enquote{test.txt} mit der Wildcard \enquote{test.*} verglichen werden und die Bibliothek würde eine Übereinstimmung melden.

\section{Parsing der Java-Dateien} 
Jede Datei, die relevant sein soll, muss anschließend weiterverarbeitet werden. Für die Bewertung der Dokumentation sind nur wenige Bestandteile relevant. Beispielsweise sind alle For-Schleifen, If-Verzweigungen und viele andere Komponenten in Methodenrümpfen nicht relevant, da diese nur mit normalen Kommentaren und nicht mit Javadoc kommentiert werden; (sie werden dennoch unstrukturiert als Zeichenkette gespeichert, damit Metriken diese Information eventuell nutzen können). Aus diesem Grund müssen die notwendigen Informationen extrahiert werden. Zudem ist es ein Ziel der Arbeit , eine Erweiterbarkeit auf andere objektorientierte Programmiersprachen zu ermöglichen. Daher müssen die Informationen in ein abstraktes Format gebracht werden, welches eine gute Annäherung für die meisten objektorientierten Programmiersprachen ist. Beispielsweise unterscheiden sich die Zugriffsmodifizierer vieler Programmiersprache, sodass eine einheitliche Schnittstelle schwer umsetzbar ist. Daher enthält die abstrakte Repräsentation nur Informationen, ob eine Komponente als öffentlich markiert ist. Dies ist sinnvoll, da öffentliche Komponenten als Teil der öffentlichen Schnittstelle eher dokumentiert werden sollten als nicht öffentliche und eine weitergehende Differenzierung kaum Vorteile bietet. In einigen Programmiersprachen wie z. B. Python gibt es keine expliziten öffentliche Komponenten, jedoch existieren de facto Standards für Bezeichner, sodass beispielsweise private Komponenten zwei Unterstriche als Präfix haben.

Außerdem werden in der abstrakten Repräsentation die Vererbung etwas vereinfacht dargestellt, indem nicht zwischen Basisklassen und Schnittstellen unterschieden wird, da es auch hier Unterschiede zwischen Programmiersprachen gibt, und die Informationen über die Vererbung, falls sie überhaupt von einer Metrik verwendet wird, vermutlich nicht so detailreich sein müsste. Zudem werden Konstruktoren als Methoden mit den Namen \enquote{constructor} und Schnittstellen als Klassen repräsentiert, da auch hier eine zu feine Spezifikation nicht notwendig sein wird.  

  

In anderen Fällen gibt es jedoch viele Gemeinsamkeiten zwischen objektorientierten Programmiersprachen; so gibt es in  allen relevanten Sprachen Klassen, Methoden und Felder, die alle einen Namen haben. Des Weiteren haben Methoden und Felder einen (Rückgabe-)Type und Methoden besitzen Parameter, die ihrerseits durch einen Namen und einen Typen definiert sind. Einige Sprachen sind zwar nicht stark typisiert, jedoch kann für nicht bekannte Datentypen ein Alias wie \enquote{Any} oder  \enquote{Object} verwendet werden.  Zudem sind viele Komponenten hierarchisch; in den meisten Sprachen können beispielsweise Klassen andere Klassen enthalten, sodass diese abstrakte Struktur diese Tatsache berücksichtigen müsste. 

Um dennoch sprachspezifische Funktionen anbieten zu können, besitzt jede Komponente ein Feld mit dem Typen \textit{ComponentMetaInformation}, das wie oben erwähnt die Information enthält, ob eine Komponente als öffentlich angesehen werden soll. Dieser Typ, welches eine Schnittstelle ist, kann von einer Klasse implementiert werden, um Parser für andere Programmiersprachen die Möglichkeit zu geben, zusätzliche sprachspezifische Informationen zu speichern. Beim Java-Parser wird diese Funktion beispielsweise genutzt, um zu speichern welche \enquote{checked} Ausnahmen eine Methode werfen kann, sodass später ein Vergleich mit der Javadoc möglich ist. Die Schnittstelle enthält nur die Anforderung, eine \textit{isPublic}-Methode anzubieten und kann daher für viele andere objektorientierte Programmiersprache Informationen speichern, die für einige Metriken eventuell nützlich sind. 
 \clearpage
 \begin{figure}[ht!]
 \begin{tikzpicture}
 \pgfmathsetlengthmacro\breite{6cm}
\pgfmathsetlengthmacro\hoehe{2.567cm}
\pgfmathsetlengthmacro\InnerSep{0.4cm}
\node[draw,
anchor=center, 
inner sep=0pt, 
minimum width=5cm, text width=6cm-\InnerSep,
align=justify,
minimum height=\hoehe
] (java) {
  \hspace*{0.1cm}\textbf{public} \textbf{class} Main\{.\newline 
     \hspace*{0.5cm}\textbf{private} \textbf{int} test(\textbf{int} a)\{\newline
    \hspace*{0.5cm}\}\newline
\hspace*{0.1cm}\}

};
\node[draw,text width=5.5cm, left = of java](python){
\textbf{class} Main:\newline
     \hspace*{0.5cm}\textbf{def} \_\_test(a:\textbf{int}): -> \textbf{int}\newline
         \hspace*{1cm}pass

};
\draw[line width=0.05cm] (python) -- (-4,-2.5) ;
\draw [line width=0.05cm](java) -- (-4,-2.5);
\draw[->,thick,line width=0.05cm] (-4,-2.5) -- (-4,-3.5);
\draw [] (-9,-3.5) rectangle(4,-16);
\begin{scope} {0,-10}
       \begin{object}[text width=8cm]{fileObj}{-4,-4}
        \instanceOf{FileComponent}
        \attribute{name = "filepath"}
        \attribute{parent = null}
        \attribute{children={[}classObj{]}}
      \end{object}
      \begin{object}[text width=8cm]{classObj}{-4,-8 }
        \instanceOf{ClassComponent}
        \attribute{name="Main"}
        \attribute{parent=fileObj}
        \attribute{children={[}methodObj{]}}
          \attribute{metaInformation=meta}
      \end{object}
    \begin{object}[text width=8cm]{methodObj}{-4,-12}
        \instanceOf{MethodComponent}
        \attribute{name="test"}
        \attribute{parent=classObj}
        \attribute{params={[}paramsObj{]}}
        \attribute{metaInformation=meta2}
        \attribute{returnType=int}
      \end{object}
      \begin{object}[text width=2.5cm]{paramObj}{2,-10}
        \attribute{name=a}
        \attribute{type=int}
      \end{object}
    \begin{object}[text width=2.5cm]{meta}{2,-6}
        \attribute{isPublic=true}
      \end{object}
    \begin{object}[text width=2.5cm]{meta2}{2.5,-14.5}
        \attribute{isPublic=false}
      \end{object}
\end{scope}
\begin{scope}[line width=0.05cm]
    \aggregation{fileObj}{}{}{classObj}{}{}
     \aggregation {classObj}{}{}{methodObj}{}{}
      \aggregation {methodObj}{}{}{paramObj}{}{}
    \aggregation {classObj}{}{}{meta}{}{}
        \aggregation {methodObj}{}{}{meta2}{}{}
\end{scope}

    
\end{tikzpicture}
     
     \caption{Objektdiagramm aus Java- und Python-Code}
     \label{fig:python_java_comp}
 \end{figure}


Abbildung \ref{fig:python_java_comp} veranschaulicht wie eine einfache Datei in die Objektstruktur umgewandelt werden kann. Dabei wird ein einfache Java-Klasse und eine semantisch äquivalente Python-Datei als Beispiel verwendet, um zu zeigen, dass aus beiden Sprachen eine gleiche Objektstruktur erzeugt werden kann. Dabei wurden zur Übersichtlichkeit einige nicht relevanten Attribute entfernt.

Das Programm in beiden Sprachen besteht aus einer öffentlichen Klasse \textit{Main} und einer privaten Methode \textit{test}, die einen Parameter \textit{a} als Ganzzahl erhält und eine Ganzzahl zurückgibt. Die höchste Hierarchieebene ist immer ein \textit{FileComponent}. Diese Datei enthält hier genau ein Kind namens \textit{classObj}, könnte aber in anderen Fällen auch mehrere Kinder (wie z.~B. Klassen enthalten). Die Klasse besitzt zudem einen Verweis auf ihren Elternteil. Außerdem enthält das \textit{classObj} eine Referenz auf MetaInformationen, die hier nur angeben, dass die Klasse öffentlich ist. In diesen MetaInformationen könnte auch die Basisklasse und andere relativ sprachspezifische Informationen enthalten. Die Klasse enthält wiederum genau die Methode als einziges Kind. Die Methode hat ebenfalls einen Verweis auf Metainformation, welche die Methode als privat markieren. Außerdem hat die Methode einen \textit{returnType} und einen Verweis auf die Liste der Parameter, die wiederum aus einen Namen und einen Datentyp bestehen.

\section{}{Repräsentation der strukturierten Kommentare}\label{chapter:structured_comments}
Neben der hierarchischen Repräsentation der einzelnen Komponenten, müssen auch die strukturierten Kommentare wie z.~B. Javadoc) geeignet in eine Datenstruktur umgewandelt werden. Wie in Kapitel \ref{chapter:javadoc}
 erläutert besteht ein strukturierter Kommentar in vielen Fällen aus zwei Teilen. Der erste Teil ist eine allgemeine Beschreibung der Komponente. Im zweiten Teil werden bestimmte Strukturen genauer erläutert. So können einzelne Parameter erklärt werden oder der Rückgabewert beschrieben werden. 
 Dieser Aufbau findet sich auch in \textit{Doxygen} \cite{doxygen} oder \textit{Docstring} \cite{docstring}, sodass diese Struktur als Grundlage genommen wird. 
 
 Ein strukturierter Kommentar besteht also aus einer generellen Beschreibung, die auch weggelassen werden kann. Anschließend folgen null bis beliebig viele Tags. Jeder Tag besteht aus einem Typ (z.~B. \enquote{@param},\enquote{@return} oder \enquote{@throws}), einem optionalen Parameter, welcher von einigen Tags benötigt wird und der Beschreibung des Tags. Die Namen der Tags sind generalisiert, dies bedeutet, dass unabhängig von der Programmiersprache der Tag zur Beschreibung eines Parameters immer \enquote{@param} heißen muss. Dies muss bei der Entwicklung eines Parsers beachtet werden. 
 
 Wird kein strukturierter Kommentar angegeben, so ist liefert der entsprechende Getter \textit{getComment} den Wert \enquote{null} zurück. 
\section{Konzeption der Metriken}
Nachdem eine Datei in ihre einzelnen Komponenten zerlegt wurde, kann die Qualität der Softwaredokumentation überprüft werden. Jede gefundene Komponente besitzt einen Verweis auf die dazugehörige Dokumentation, die bei Nichtvorhandensein auch null sein kann. Anhand dieser Referenz kann geprüft werden, ob die Softwaredokumentation der Komponente ausreichend ist. Zur Bewertung der Dokumentation gibt es verschiedene  Möglichkeiten. Beispielsweise könnte überprüft werden, ob eine Komponente dokumentiert oder undokumentiert ist. Eine weitere Möglichkeit wäre es die Verständlichkeit der Dokumentation zu prüfen. Jedes dieser Vorgehen basiert auf eine Metrik, die auf wissenschaftliche Studien beruht oder zumindest plausibel ist. In diesem Abschnitt wird ein Konzept erläutert, um eine Metrik zu implementieren. Anschließend wird beschrieben, wie die Ergebnisse jeder Metrik zusammengefasst werden, um ein Endresultat zu erhalten. 

\subsection{Implementation einer Metrik}\label{chapter:metric_impl}
Damit eine Metrik die Dokumentation bewerten kann, benötigt sie Zugriff auf die Komponente. Außerdem muss sie ihr Ergebnis irgendwie veröffentlichen bzw. zwischenspeichern, damit es später weiterverarbeitet werden kann. Des Weiteren ist nicht jede Metrik mit jede Komponente kompatibel. Eine Metrik, die überprüft, ob jeder Methodenparameter dokumentiert ist (vgl. Kapitel \ref{chapter:method_doc}), kann mit anderen Komponentenarten weniger anfangen. Zudem sollte es die Möglichkeit geben, das Verhalten einer Metrik mittels Parameter anzupassen, damit die Metrik konfigurierbar bleibt. Zuletzt sollte eine Metrik bei der Bewertung auch begründen, warum die Dokumentation einer Komponente nicht ausreichend ist. Außerdem soll der Benutzer des Tools selbst auswählen können, welche Metriken angewendet werden sollen, da nicht jede Metrik immer sinnvoll ist. 

Eine Möglichkeit, diese Anforderung für eine Metrik umzusetzen, wäre die Verwendung einer Methode pro Metrik, welche die Komponente und die Parameter als Eingabe erhält und daraus die Bewertung und eventuelle Begründung ermittelt und zurückgibt. Allerdings ist dieser Ansatz sehr prozedural; es gibt beispielsweise keine Kapselung zwischen den Metriken.  

Ein anderer Ansatz, der hier auch gewählt wird, ist es, jede gewünschte Metrik als Klasse zu implementieren. Jede implementierte Metrik kann somit die notwendigen Berechnungen abgekapselt von anderen Metriken erledigen, was die Wartbarkeit verbessert. Um trotzdem für eine einheitliche Schnittstelle zu sorgen, muss jede zu implementierende Metrik von einer abstrakten Basisklasse \textit{(DocumentationAnalysisMetric)} erben,, welche die Implementation bestimmter Methoden vorschreibt, sodass ein Benutzer der Metrik auch ohne Wissen über die Interna der Metrik diese verwenden kann. Die Methode \textit{shallConsider} überprüft, ob eine Komponente für diese Metrik ist geeignet ist und gibt dementsprechend einen Wahrheitswert zurück. Die Methode \textit{analyze} führt anschließend die Bewertung durch.

Zudem kann ein Objekt, das eine Metrik repräsentiert und daher von \textit{DocumentationAnalysisMetric} erbt (Metrikobjekt), Parameter besitzen, welche spezifische Eigenschaften der Metrik modifizieren kann. Diese Eigenschaft werden sehr abhängig von der Metrik sein, sodass eine einheitliche Schnittstelle nur schwer umsetzbar ist. Daher werden die Parameter als Datentyp \textit{any} übergeben, sodass es keine Typüberprüfung gibt. Alternativ wäre eine assoziative Liste möglich, bei dem ein Parametername als Zeichenketter ein Wert zugeordnet wird, aber auch hier könnte kein Überprüfung eines Datentypes vorgenommen werden. 

Eine  weitere Voraussetzung für ein Metrikobjekt ist ein eindeutiger Name. Dadurch kann die gleiche Metrik mit unterschiedlichen Parametern verwendet werden. Außerdem wird so eine Zuordnung von Gewichten vereinfacht. Standardmäßig besteht dieser eindeutige Name aus dem Namen der implementierten Metrik gefolgt von einem Unterstrich und einer fortlaufenden Nummer.
\subsubsection{Bewertung der Dokumentation}
Die Methode \textit{analyze} muss eine Bewertung darüber abgeben, ob die Qualität der Dokumentation ausreichend ist. Für die Repräsentation dieser Bewertung gibt es viele Möglichkeiten, allerdings ist eine numerische Bewertung mittels einer Intervallskala am sinnvollsten, da so der arithmetische Mittelwert, der Median etc. berechnet werden kann, was für die Bildung des Gesamtergebnisses wichtig ist.
Die numerische Bewertung soll eine Aussage über die Dokumentationsqualität liefern. Eine Bewertung von 0 steht für eine sehr schlechte bis nicht existente Dokumentation und die Bewertung 100 steht für eine exzellente Dokumentation, sodass die Bewertung sich als Prozent lesen lassen kann. Das Ergebnis einer implementierten Metrik sollte diesen Wertebereich nicht verlassen, da eine Fehlerbehandlung nicht implementiert ist. Bei Metriken, die per Design schon eine prozentualen Wert zurückgeben (z.~B. Kapitel \ref{chapter:metrics_simple_comment}) wird diese Vorgabe stets eingehalten. Bei anderen Metriken (z.~B. \ref{chapter:metrics_flesh}) sollte eine mathematische Funktion gefunden werden, die das Ergebnis der Metrik auf den Wertebereich 0 bis 100 abbildet. Die genaue Umsetzung hängt von der Metrik ab. In jedem Falle sollte es für eine Metrik Ergebnisse geben, die auf eine gute bzw. schlechte Dokumentation hindeuten, damit diese auf 100 bzw. 0 abgebildet werden können. Nur durch diese Einschränkung auf einen fixen Wertebereich ist es möglich, den Mittelwert, Median etc. zu bilden und so eine Vergleichbarkeit zu ermöglichen. 

\subsection{Ergebnis der Metrik verarbeiten}

Das berechnete Ergebnis einer Komponente muss nun gespeichert werden, damit es später ausgewertet werden kann. Dazu wird ein \textit{MetricResult}-Objekt erstellt, welches das im vorherigen Unterabschnitt berechnet Ergebnis enthält. Außerdem werden hier eventuelle Begründungen und Hinweise gespeichert, die dem Anwender dabei unterstützen, die Qualität der Dokumentation zu verbessern. Jede Begründung enthält den Dateipfad der betroffenen Datei, die bemängelte Komponente und die Zeilennummer, sodass der Benutzer die problematische Stelle schnell finden kann. Zuletzt wird außerdem eine Zeichenkette gespeichert, die später für die Gewichtung relevant ist. 

Für die Speicherung des Objekt gibt es zwei Möglichkeiten. Die erste Möglichkeit wäre es, dass die \textit{analyze}-Methode das \textit{MetricResult}-Objekt einfach zurückgibt, sodass der Aufrufer damit arbeiten kann. Bei der zweiten Möglichkeit wird das Ergebnis einem anderen Objekt übergeben, der dann die Weiterverarbeitung vornimmt. Dies hat den Vorteil, dass eine Metrik kein Ergebnis zurückliefern muss, wenn es kein sinnvolles Ergebnis berechnen kann. Bei einem Rückgabewert müsste ansonsten ein ungültiger Wert wie z.~B. \textit{null} vereinbart werden. Außerdem kann eine Metrik auch mehrere Resultate speichern, was bei komplexeren Komponenten interessant in Betracht gezogen werden könnte. Dieses weitere Objekt ist ein  \textit{MetricResultBuilder}, der wie in nächsten Unterabschnitt beschrieben, die Softwaredokumentationsqualität jeder Komponente sammelt und daraus ein Gesamtergebnis berechnet.  
\subsection{Resultate der Metriken anwenden}
Da in einer Datei mehrere Komponenten durchaus Standard sind (eine Klasse mit einer enthaltener Methode zählt bspw. als zwei Komponenten), müssen die Bewertungen jeder Komponente passend aggregiert werden. Dazu wird dem \textit{MetricResultBuilder} jedes Ergebnis mittels der \textit{processResult}-Methode mitgeteilt, welches das Ergebnis in einer Liste speichert. Wenn alle Metriken verarbeitet sind, wird daraus ein Gesamtresultat gebildet. Dies geschieht durch die Methode \textit{getAggregratedResult}. Dabei wird standardmäßig ein arithmetischer Mittelwert gebildet. Durch Ableitung kann diese Methode überschrieben werden, um den Median oder eine Gewichtung bei der Bildung des Gesamtergebnisses zu verwenden  sodass die Wahl des Algorithmus flexibel bleibt . Ein \textit{ResultBuilder} basiert auf dem Vorbild des Design-Patterns \enquote{Builder} aus \cite[S.139-149]{gamma2015design}, da es aus einzelnen Metrikresultaten ein vollständiges Metrikergebnis baut.

Abbildung \ref{fig:metrics_apply} zeigt schematisch, wie aus verschiedenen Dateien ein Gesamtergebnis ausgebaut wird. Hier werden zwei Dateien, die zur Vereinfachung gleich aufgebaut sind, und zwei exemplarische Metriken verwendet. Links unten wird die erste Datei (rot) durch die ersten Metrik bewertet. Diese Datei besteht aus einer Methode und einer Klasse. Die Ergebnisse beider Komponenten werden durch einen dafür bestimmten \textit{MetricResultBuilder} zu einem Gesamtergebnis verarbeitet. Dabei kann beispielsweise das Ergebnis der Methode höher bewertet werden als das der Klasse.  Dieses Ergebnis kann dann durch die Metrik verarbeitet werden.  Auch die zweite Metrik bewertet die gleiche Datei mit ihren beiden Komponenten auf die gleiche Art und Weise, sodass von beiden Metriken ein Ergebnis vorliegt. Diese Ergebnisse werden von einem weiteren\textit{MetricResultBuilder} verarbeiten, der die Ergebnisse je nach Metrik gewichten kann. Das Ergebnis dieses Vorgangs wird von einem anderen \textit{MetricResultBuilder} verarbeitet, welcher dieses und das Ergebnis der zweiten Datei (grün), das analog ermittelt wird, verarbeitet und und eine Gewichtung nach Dateipfad vornehmen könnte. Das Gesamtergebnis oben in der Mitte ist dann Maß für die Entscheidung, ob die Dokumentationsqualität als ausreichend angesehen wird. 
\begin{figure}[ht!]
\fontsize{7}{10}\selectfont
    \centering
\includesvg[scale=0.8]{figures/chapter3/metrics.svg}
    \caption{Anwendung der Metriken}
    \label{fig:metrics_apply}
\end{figure}
 
\subsection{Zuordnung der Gewichte}\label{chapter_weights_assign}
Für einige Algorithmen muss eine Gewichtung vorgenommen werden, um bestimmte Ergebnisse besser oder schlechter zu bewerten. Dazu muss jedes Teilergebnis eine Gewicht zugeordnet werden. 

Ein Teilergebnis kann hier entweder von einer einzelnen Metrik, einer Komponente oder einer einzelnen Datei produziert werden. Durch die Gewichtung einer Metrik können bestimmten Metriken einen größeren Einfluss auf das Gesamtergebnis haben, wenn diese als vertrauenswürdiger empfunden werden. Durch die Gewichtung von Dateien kann beispielsweise die öffentliche \ac{API} einen größeren Einfluss auf die Bewertung nehnem, da diese Komponenten bzw. Dateien sehr kritisch sein können und daher gut verstanden werden müssen. Auch eine Gewichtung nach Komponenten kann in bestimmten Situationen sinnvoll sein. So kann beispielsweise argumentiert werden, dass Methoden, die durch ihre Parameter, Rückgabewerte und geworfene Ausnahmen komplexer als Felder sind, höher gewichtet werden sollen.

Die Zuordnung der Gewichte erfolgt über einen \textit{WeightResolver}, welches eine Schnittstelle anbietet, um einen Bezeichner auf ein Gewicht abzubilden. Bei dem eindeutigen Namen eines Metrikobjektes kann hierfür eine assoziative Liste verwendet werden. Auch bei Komponenten, die durch ihren Klassennamen (wie z.~B. \textit{ClassComponent}) repräsentiert werden, ist  eine solche assoziative Liste sinnvoll, da es nur eine begrenzte Anzahl an Metriken bzw. Komponenten existieren kann.
Für Dateipfade ist dies allerdings nicht praktikabel, da es eine Vielzahl an Dateien geben kann. Stattdessen können hier ähnlich wie bei der Filterung von Dateien in Kapitel \ref{chapter:traversing} Wildcard-Patterns verwendet werden. Eine assoziative Liste kann dazu jedes Wildcard-Patterns und das dazugehörige Gewicht speichern. Bei einer Abfrage kann das Gewicht des ersten Eintrages zurückgegeben werden, bei dem der Dateipfad mit dem Wildcard-Patterns kompatibel ist. Dies ermöglicht es, ganze Verzeichnisse oder Dateien mit bestimmten Namen stärker zu gewichten. 

Bei der Suche nach dem passenden Gewicht zu den Dateien und Komponenten kann es vorkommen, dass das passende Gewicht nicht gefunden wird. Um hierdurch entstehende Probleme zu vermeiden, wird ein dann ein Srandardwert genommen (z.~B. 1). Bei den Metriken sollte dieses Problem nicht auftreten, da der Nutzer jede zu verwendende Metrik selbst definieren kann und somit das Gewicht auch setzen muss.

Jedes \textit{MetricResult}-Objekt enthält eine Zeichenkette, die den Erzeuger des Ergebnisses repräsentiert; so kann jedes Einzelresultat passend gewichtet werden. Bei der Bildung eines Gesamtergebnis aus einzelnen Teilresultaten mittels der Methode \textit{getAggregatedResult} muss das neue \textit{MetricResult}-Objekt ebenfalls ein Erzeuger besitzen. Dieser Erzeuger wird vom Anwender der Funktion übergeben. Werden beispielsweise die Ergebnisse mehrere Metriken aber einer Datei zusammengefasst, so ist der neue Erzeuger der Dateipfad. Bei dem endgültigen Gesamtergebnis ist eine Gewichtung nicht mehr sinnvoll, daher kann dann eine leere Zeichenkette übergeben werden.


\subsection{Sprachspezifische Informationen für Metriken}\label{chapter:langSpec}
Da das Tool für möglichst viele objektorientierte Programmiersprachen konzipiert werden soll, muss eine Generalisierung erfolgen, da jede Sprache ihre Eigenheiten hat und möglicherweise besondere Funktionen anbietet, die nur schwer in einem abstrakten Format zu bringen sind.

Nichtsdestotrotz können solche sprachspezifischen Eigenheiten auch in der Dokumentation erwähnt werden. daher ist es eine sinnvolle Idee, dass Metriken auch diese Besonderheiten benutzen, um ein genaueres Bild der Dokumentationsqualität zu erfahren, ohne jedoch zu wissen, welche Programmiersprache gerade analysiert wird. Beispielsweise können die Checked-Ausnahmen in Java mit den Informationen in der Javadoc verglichen werden. Des Weiteren können. Komponenten einer Schnittstelle oder abstrakten Klasse, die es nicht jeder Programmiersprache gibt, stärker geprüft werden und bei mangelhafter Dokumentation stärker gewichtet werden, da dort die Erfüllung eines Vertrages wichtig ist und daher eine gute Dokumentation wichtiger ist. Auch eine feinere Abstufung je nach Zugriffsmodifizierer einer Komponente wäre so möglich, sodass eine undokumentierte öffentliche Komponente schelchter bewertet wird, als eine Komponente mit dem Zugriffsmodifizierer  \textit{protected}. 

Um solche sprachspezifischen Analysen zu erlauben, besitzt jede Metrik Zugriff auf ein Objekt der Klasse \textit{LanguageSpecificHelper}. Wenn eine neue Programmiersprache hinzugefügt werden soll, kann von dieser geerbt werden. In der Klasse \textit{LanguageSpecificHelper} sidn bereits einige Methoden definiert, die einigen Metriken bei der Bewertung helfen. So bewertet die Methode \textit{rateDocumentationCompatibility}, ob die Dokumentation alle sprachspezifischen Informationen erläutert (z.~B. \textit{@throws}). Die Methode \textit{shallConsider} kann genutzt werden, um überschriebene Methoden zu ignorieren. 

Um eigene Methoden hinzuzufügen, muss diese in der Basisklasse definiert werden. Diese Methode sollte in der Basisklasse keine Aktionen durchführen, sondern entweder gar nichts tun oder Rückgabewerte haben, die keinen Einfluss auf Metriken haben. Anschließend kann diese Methode in einer sprachspezifischen abgeleiteten Klasse der Basisklasse korrekt implementiert werden. Die entsprechende Methode kann dann durch Änderung des Quellcodes der Metrik an den passenden Stellen von der Metrik verwendet werden. So kann beispielsweise das Resultat einer Metrik modifiziert werden oder weitere Ergebnisse mittels des \textit{MetricResultBuilder} angefügt werden.

\subsection{Ignorieren bestimmter Kommentare}

Unter Umständen kann es sinnvoll sein, bestimmte Komponenten bei der Bewertung auszulassen, weil sie beispielsweise noch nicht vollständig implementiert sind, in einer nicht-englischen Sprache dokumentiert sind oder ein anderer gewichtiger Grund existiert. Für diesen Fall kann die allgemeine Beschreibung der Dokumentation einer Komponente  den Begriff \enquote{\%ignore\_this\% } oder \enquote{\%ignore\_node\%} enthalten. Bei Ersterem wird nur diese Komponente ignoriert und als nicht existent betrachtet. Bei Zweiterem werden sowohl diese Komponente als auch alle Kinder dieser Komponente ignoriert, sofern sie existieren



