\section{Einleitung}
Zur Umsetzung des Tools wurde das Projekt in verschiedene Arbeitspakete zerlegt, wobei jeder Arbeitspaket auf den Vorgänger aufbaut. Diese Arbeitspakete und die damit verbundenen Herausforderungen werden nun erläutert.  

  
\section{Arbeitspakete}
\subsection{Traversierung aller relevanten Dateien} 

Softwareprojekte bestehen aus Hunderten von Dateien, die nicht alle Quellcode enthalten. Beispielsweise gehören Konfigurationsdateien, Ressourcedateien wie Bilder oder binäre Dateien zu den Dateien, bei denen eine Analyse der Softwaredokumentation nicht zweckmäßig wäre oder sogar zu Fehlern führen könnte. Daher ist es sinnvoll bestimmte Dateien bei der Analyse auszuschließen beziehungsweise nur bestimmte Dateien zu betrachten. Beim JavadocEvaluator wird hierzu die NPM-Bibliothek Minimatch \cite{@tComment:TestingJavadocCommentstoDetectComment-CodeInconsistencies} verwendet, die es ermöglicht Dateinamen mit Wildcard-Patterns zu vergleichen. Zum Beispiel könnte der Dateiname \enquote{test.txt} mit der Wildcard \enquote{test.*} verglichen werden und die Bibliothek würde eine Übereinstimmung melden.  

  

In der Konfigurationsdatei können solche Wildcard-Patterns sowohl für Dateien, die zwingend analysiert werden müssen, als auch Dateien, die in jeden Fall ausgeschlossen werden müssen, definiert werden. So kann der Benutzer des Tools beispielsweise nur Dateien mit der Dateiendung \textit{.java} analysieren, was in der Standardeinstellung auch so passiert, und Dateien aus bestimmten Verzeichnissen ignorieren, weil sie Unit-Tests enthalten.   

  

\subsection{Parsing der Java-Dateien} 

Jede Datei, die laut den vorherigen Arbeitspaket relevant sein soll, muss anschließend weiterverarbeitet werden. Für die Bewertung der Dokumentation sind nur wenige Bestandteile relevant. Beispielsweise sind alle For-Schleifen, If-Verzweigungen und viele andere Komponenten in Methodenrümpfen nicht relevant, da diese nur mit normalen Kommentaren und nicht mit Javadoc kommentiert werden.  Aus diesen Gründen müssen die notwendigen Informationen extrahiert werden. Zudem ist es ein Ziel des Tools, eine Erweiterbarkeit auf andere objektorientierte Programmiersprachen zu ermöglichen. Daher müssen die Informationen in ein abstraktes Format gebracht werden, welches eine gute Annäherung für die meisten objektorientierten Programmiersprachen ist. Beispielsweise unterscheiden sich die Zugriffsmodifizierer vieler Programmiersprache, sodass eine einheitliche Schnittstelle schwer umsetzbar ist. Daher enthält die abstrakte Repräsentation nur Informationen, ob eine Komponente als öffentlich markiert ist. Dies ist sinnvoll, da öffentliche Komponenten als Teil der öffentlichen Schnittstelle eher dokumentiert werden sollten als nichtöffentliche und eine weitergehende Differenzierung kaum Vorteile bietet. In einigen Programmiersprachen wie z. B. Python gibt es keine expliziten öffentliche Komponenten, jedoch existieren de-facto Standards für Bezeichner, sodass beispielsweise private Komponenten zwei Unterstriche als Präfix haben.

Außerdem werden in der abstrakten Repräsentation die Vererbung etwas vereinfacht dargestellt, indem nicht zwischen Basisklassen und Schnittstellen unterschieden wird, da es auch hier Unterschiede zwischen Programmiersprachen gibt, und die Informationen über die Vererbung, falls sie überhaupt von einer Metrik verwendet wird, vermutlich nicht so detailreich sein müsste. Zudem werden Konstruktoren als Methoden mit den Namen \enquote{constructor} und Schnittstellen als Klassen repräsentiert, da auch hier eine zu feine Spezifikation nicht notwendig sein wird.  

  

In anderen Fällen gibt es jedoch viele Gemeinsamkeiten zwischen objektorientierten Programmiersprachen; so gibt es in  allen relevanten Sprachen Klassen, Methoden und Felder, die alle einen Namen haben. Des Weiteren haben Methoden und Felder einen (Rückgabe-)Type und Methoden besitzen Parameter, die ihrerseits durch einen Namen und einen Typen definiert sind. Einige Sprachen sind zwar nicht stark typisiert, jedoch kann für nicht bekannte Datentypen ein Alias wie \enquote{Any} oder  \enquote{Object} verwendet werden.  Zudem sind viele Komponenten hierarchisch; in den meisten Sprachen können beispielsweise Klassen andere Klassen enthalten, sodass diese abstrakte Struktur diese Tatsache berücksichtigen müsste. 

Um dennoch sprachspezifische Funktionen anbieten zu können, besitzt jede Komponente ein Feld mit dem Typen \textit{ComponentMetaInformation}, das wie oben erwähnt die Information enthält, ob eine Komponente als öffentlich angesehen werden soll. Dieser Typ, welches eine Schnittstelle ist, kann von einer Klasse implementiert werden, um Parser für andere Programmiersprachen die Möglichkeit zu geben, zusätzliche sprachspezifische Informationen zu speichern. Beim Java-Parser wird diese Funktion beispielsweise genutzt, um zu speichern welche \enquote{checked} Ausnahmen eine Methode werfen kann, sodass später ein Vergleich mit der Javadoc möglich ist. Die Schnittstelle enthält nur die Anforderung, eine \textit{isPublic}-Methode anzubieten und kann daher für viele andere objektorientierte Programmiersprache Informationen speichern, die für einige Metriken eventuell nützlich sind. 

Um die Java-Dateien zu parsen, wird die Bibliothek ANTLR4 verwendet, die kostenlos verfügbar ist. Diese Bibliothek kann nicht nur viele Programmiersprachen, sondern auch selbst geschriebene Sprachen parsen, sofern eine passende Grammatik verfügbar ist. Daneben gibt es noch Bibliotheken wie Bison, PEG.JS, die ebenfalls eine ähnliche Aufgabe erfüllen können. Allerdings gibt es bereits vordefinierte Grammatiken für ANTLR4, sodass ein erheblicher Aufwand gespart werden konnte.  

Eine Grammatik besteht immer aus einer Lexer-Grammatik und eine Parser-Grammatik, die jeweils vom Lexer bzw. Parser in dieser Reihenfolge verarbeitet werden, um so eine Baumstruktur einer Quellcodedatei zu erhalten. Der Lexer erstellt aus einer Quellcodedatei eine Liste von Tokens. Dabei ist ein Token eine Gruppierung von einem oder mehreren Zeichen, die eine weitergehende Bedeutung haben. Beispielsweise können Schlüsselwörter eine Programmiersprache oder Operatoren als Token klassifiziert. Diese Tokens sind grundsätzlich kontextunabhängig, das heißt für für die gleiche Zeichenkette wird der das gleiche Token verwendet. Ausnahmen gibt es beispielsweise, wenn ein Schlüsselwort  als ein Token betrachtet wird und sie von Anführungszeichen umschlossen wird, sodass das Schlüsselwort als String betrachtet wird \todo{reword}. Durch die Kontextunabhängigkeit wird ein Schlüsselwort einer Programmiersprache auch mit dem gleichen Token bezeichnet, auch wenn ein Schlüsselwort mehrere Bedeutungen haben kann.  


Der nächste Schritt ist das eigentliche Parsen. Dabei werden die im vorherigen Schritt generierten Tokens genommen und in eine Baumstruktur umgewandelt, wobei hier der Kontext die entscheidende Rolle spielt. \todo{write more}.


Für die Programmiersprache Java steht bereits eine Grammatik, die auf Github unter der BSD-Lizenz angeboten wird ist, zur Verfügung, allerdings ignoriert sie die Kommentare. daher mussten einige Änderungen sowohl am Lexer als auch am Parser vorgenommen werden. Im Lexer werden standardmäßig alle Tokens in einem Kommentar in einen versteckten Kanal gespeichert, was dazu führt, dass diese Tokens vom Parser ignoriert werden. Daher wurde dieses Verhalten durch Definition eines neuen Tokens so geändert, dass Javadoc-Kommentare auch vom Parser verarbeitet werden können, aber mehrzeilige und einzeilige Kommentare weiterhin ignoriert werden. Einzeilige Kommentare sind hier nicht relevant, da sie kein Javadoc enthalten. Mehr Änderungen mussten an der entsprechenden Parser-Datei \textit{JavaParser.g4} durchgeführt werden.  Tabelle \ref{tab:parser_changes} listet alle Änderungen an der Parserdatei auf;
\begin{table}[]
    \centering
    \begin{tabular}{m{0.75cm}|m{4cm}|m{10cm}}
        Zeile & Änderung & Begründung \\
         \hline
        116 & Deklaration Kommentar & Hier wird ein mehrzeiliger Kommentar definiert, dies ist hier ein Alias für den Token \textit{COMMENT}\\
        \hline
        128 & \textit{comment} als möglicher Präfix in Klassenmember & Hier wird dem Parser mitgeteilt, dass ein Bestandteil einer Klasse wie z. B. eine Methode einen Javadoc-Kommentar besitzen kann\\
        \hline
        47 & \textit{comment} als möglicher Präfix vor Datentyp & Hier wird dem Parser mitgeteilt, dass ein Datentyp (Klasse, Schnittstelle etc. ) einen Javadoc-Kommentar haben kann \\
        \hline
        404 & Zulassung von mehrzeiligen Kommentaren an anderen Stellen & Da Javadoc-Kommentare an beliebigen Stellen auftauchen können, auch wenn es nicht empfohlen wird und keinen Mehrwert bietet, wird hier sichergestellt, dass solche Kommentare nicht zu Warnungen oder Fehler von ANTLR4 führen. Diese Javadoc-Kommentare werden nichtsdestotrotz später ignoriert.
    \end{tabular}
    \caption{Änderungen an der Parserdatei}
    \label{tab:parser_changes}
\end{table}

\subsection{Metriken anwenden}
Als nächsten Schritt müssen auf die Informationen bezüglich der Dokumentation, die der Parser im vorherigen Schritt gesammelt hat, geeignet ausgewertet werden. Dazu können verschiedene Metriken verwenden, die im Späteren genauer erläutert werden. Eine Metrik analysiert genau eine Komponente und die dazugehörige Dokumentation. Jede Metrik muss eine numerische Bewertung  abgeben, die eine Aussage über die Dokumentationsqualität trifft. Hier wurde eine 0 für sehr schlechte bis nicht existente Dokumentation und 100 für eine exzellente Dokumentation gewählt, sodass die Bewertung sich als Prozent lesen lassen kann. Grundsätzlich sind aber auch andere Zahlenbereiche möglich.
Da in einer Datei mehrere Komponenten durchaus Standard sind, müssen die Bewertungen jeder Komponente passend aggregiert werden. Die einfache Lösung ist einen einfachen arithmetischen Mittel zu verwenden. Auch eine gewichteter Mittelwert ist denkbar und wäre vor allem geeignet, um die Bewertungen mehrerer Metriken passend zu verknüpfen. Natürlich sollte sich die Evaluation nicht nur auf eine Datei beschränken, sondern auch für das gesamte Projekt gelten. Daher müssen die Einzelresultate auf eine ähnliche Art und Weise zu einem Gesamtresultat aggregiert werden. Dieses Gesamtresultat soll dann der Hauptfaktor bei der Entscheidung sein, ob die Softwaredokumentation ausreichend ist. 

Zudem können Metriken Warn- oder Fehlermeldungen werfen, um den Anwender des Programms auf schlechte Dokumentation aufmerksam zu machen. Diese Fehlermeldung enthalten auch den Komponentennamen und die Zeilennummern, damit der Programmierer schnell auf die verwendete Komponente zugreifen kann. 
