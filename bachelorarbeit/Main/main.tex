\section{Einleitung}
Zur Umsetzung des Tools wurde das Projekt in verschiedene Arbeitspakete zerlegt, wobei jedes Arbeitspaket auf den Vorgänger aufbaut. Diese Arbeitspakete und die damit verbundenen Herausforderungen werden nun erläutert.

  
\section{Arbeitspakete}
\subsection{Traversierung aller relevanten Dateien} 
Softwareprojekte bestehen aus Hunderten von Dateien, die nicht alle Quellcode enthalten. Beispielsweise gehören Konfigurationsdateien, Ressourcedateien wie Bilder oder binäre Dateien zu den Dateien, bei denen eine Analyse der Softwaredokumentation nicht zweckmäßig wäre oder sogar zu Fehlern führen könnte. Daher ist es sinnvoll, bestimmte Dateien bei der Analyse auszuschließen beziehungsweise nur bestimmte Dateien zu betrachten. Beim JavadocEvaluator wird hierzu die NPM-Bibliothek Minimatch \cite{Minimatch} verwendet, die es ermöglicht, Dateinamen mit Wildcard-Patterns zu vergleichen. Zum Beispiel könnte der Dateiname \enquote{test.txt} mit der Wildcard \enquote{test.*} verglichen werden und die Bibliothek würde eine Übereinstimmung melden.

  

In der Konfigurationsdatei können solche Wildcard-Patterns sowohl für Dateien, die zwingend analysiert werden müssen, als auch Dateien, die in jeden Fall ausgeschlossen werden müssen, definiert werden. So kann der Benutzer des Tools beispielsweise nur Dateien mit der Dateiendung \textit{.java} analysieren, was in der Standardeinstellung auch so passiert, und Dateien aus bestimmten Verzeichnissen ignorieren, weil sie Unit-Tests enthalten.   

  

\subsection{Parsing der Java-Dateien} 
Jede Datei, die laut den vorherigen Arbeitspaket relevant sein soll, muss anschließend weiterverarbeitet werden. Für die Bewertung der Dokumentation sind nur wenige Bestandteile relevant. Beispielsweise sind alle For-Schleifen, If-Verzweigungen und viele andere Komponenten in Methodenrümpfen nicht relevant, da diese nur mit normalen Kommentaren und nicht mit Javadoc kommentiert werden; (sie werden dennoch unstrukturiert als Zeichenkette gespeichert, damit Metriken diese Information eventuell nutzen können). Aus diesen Gründen müssen die notwendigen Informationen extrahiert werden. Zudem ist es ein Ziel des Tools, eine Erweiterbarkeit auf andere objektorientierte Programmiersprachen zu ermöglichen. Daher müssen die Informationen in ein abstraktes Format gebracht werden, welches eine gute Annäherung für die meisten objektorientierten Programmiersprachen ist. Beispielsweise unterscheiden sich die Zugriffsmodifizierer vieler Programmiersprache, sodass eine einheitliche Schnittstelle schwer umsetzbar ist. Daher enthält die abstrakte Repräsentation nur Informationen, ob eine Komponente als öffentlich markiert ist. Dies ist sinnvoll, da öffentliche Komponenten als Teil der öffentlichen Schnittstelle eher dokumentiert werden sollten als nicht öffentliche und eine weitergehende Differenzierung kaum Vorteile bietet. In einigen Programmiersprachen wie z. B. Python gibt es keine expliziten öffentliche Komponenten, jedoch existieren de facto Standards für Bezeichner, sodass beispielsweise private Komponenten zwei Unterstriche als Präfix haben.

Außerdem werden in der abstrakten Repräsentation die Vererbung etwas vereinfacht dargestellt, indem nicht zwischen Basisklassen und Schnittstellen unterschieden wird, da es auch hier Unterschiede zwischen Programmiersprachen gibt, und die Informationen über die Vererbung, falls sie überhaupt von einer Metrik verwendet wird, vermutlich nicht so detailreich sein müsste. Zudem werden Konstruktoren als Methoden mit den Namen \enquote{constructor} und Schnittstellen als Klassen repräsentiert, da auch hier eine zu feine Spezifikation nicht notwendig sein wird.  

  

In anderen Fällen gibt es jedoch viele Gemeinsamkeiten zwischen objektorientierten Programmiersprachen; so gibt es in  allen relevanten Sprachen Klassen, Methoden und Felder, die alle einen Namen haben. Des Weiteren haben Methoden und Felder einen (Rückgabe-)Type und Methoden besitzen Parameter, die ihrerseits durch einen Namen und einen Typen definiert sind. Einige Sprachen sind zwar nicht stark typisiert, jedoch kann für nicht bekannte Datentypen ein Alias wie \enquote{Any} oder  \enquote{Object} verwendet werden.  Zudem sind viele Komponenten hierarchisch; in den meisten Sprachen können beispielsweise Klassen andere Klassen enthalten, sodass diese abstrakte Struktur diese Tatsache berücksichtigen müsste. 

Um dennoch sprachspezifische Funktionen anbieten zu können, besitzt jede Komponente ein Feld mit dem Typen \textit{ComponentMetaInformation}, das wie oben erwähnt die Information enthält, ob eine Komponente als öffentlich angesehen werden soll. Dieser Typ, welches eine Schnittstelle ist, kann von einer Klasse implementiert werden, um Parser für andere Programmiersprachen die Möglichkeit zu geben, zusätzliche sprachspezifische Informationen zu speichern. Beim Java-Parser wird diese Funktion beispielsweise genutzt, um zu speichern welche \enquote{checked} Ausnahmen eine Methode werfen kann, sodass später ein Vergleich mit der Javadoc möglich ist. Die Schnittstelle enthält nur die Anforderung, eine \textit{isPublic}-Methode anzubieten und kann daher für viele andere objektorientierte Programmiersprache Informationen speichern, die für einige Metriken eventuell nützlich sind. 

Um die Java-Dateien zu parsen, wird die Bibliothek ANTLR4 verwendet, die kostenlos verfügbar ist. Diese Bibliothek kann nicht nur viele Programmiersprachen, sondern auch selbst geschriebene Sprachen parsen, sofern eine passende Grammatik verfügbar ist. Daneben gibt es noch Bibliotheken wie Bison, PEG.JS, die ebenfalls eine ähnliche Aufgabe erfüllen können. Allerdings gibt es bereits vordefinierte Grammatiken für ANTLR4, sodass ein erheblicher Aufwand gespart werden konnte.  

Zuvor wurde bereits versucht, die Java-Dateien mittels der NPM-Bibliothek \enquote{java-parser} \cite{Java-parser} zu parsen. Dies hätte den Vorteil, dass das Parsen, welches nicht der Hauptfokus dieser Bachelorarbeit ist, ausgelagert wird und so Fehler vermieden wird. Allerdings konnte diese Bibliothek nicht die Verbindung zwischen einer Komponente und der dazugehörigen Javadoc herstellen, sodass eine Benutzung dieser Bibliothek die Arbeit deutlich erschwert hätte. Daher fiel die Wahl auf \textit{ANTLR4}. Außerdem wurde zunächst versucht eine eigene Grammatik für Java zu schreiben, da die originale Grammatik für Java sämtliche Kommentare ignoriert. Da dies jedoch deutlich komplizierter war als erwartet, wurde dennoch entschieden, die von vielen Entwicklern geschriebene Grammatik zu verwenden (siehe unten).

Eine Grammatik besteht immer aus einer Lexer-Grammatik und eine Parser-Grammatik, die jeweils vom Lexer bzw. Parser in dieser Reihenfolge verarbeitet werden, um so eine Baumstruktur einer Quellcodedatei zu erhalten. Der Lexer erstellt aus einer Quellcodedatei eine Liste von Tokens. Dabei ist ein Token eine Gruppierung von einem oder mehreren Zeichen, die eine weitergehende Bedeutung haben. Beispielsweise können Schlüsselwörter einer Programmiersprache oder Operatoren als Token klassifiziert werden. Diese Tokens sind grundsätzlich kontextunabhängig, das heißt für die gleiche Zeichenkette wird der das gleiche Token verwendet. Ausnahmen gibt es beispielsweise, wenn ein Schlüsselwort von Anführungszeichen umschlossen wird, sodass das Schlüsselwort als String betrachtet wird \todo{reword}. Durch die Kontextunabhängigkeit wird ein Schlüsselwort einer Programmiersprache auch

Listening \ref{lst:lexer_example} zeigt zur Verdeutlichung der Syntax eine Zeile aus der Lexer-Grammatik. 
		\begin{figure} [htbp]
			\lstinputlisting
			[caption={Beispielhafte Syntax vom Lexer},
			label={lst:lexer_example},
			captionpos=b, basicstyle=\footnotesize, tabsize=2, showstringspaces=false,  numbers=left]
			{figures/lexer_example.g4}
		\end{figure}

Eine Zeile in der Lexer-Grammatik beginnt mit den Namen eines Tokens, gefolgt von einem Doppelpunkt und dann den Zeichen, die dieses Token beschreiben. Hier wird ein Javadoc-Kommentar definiert, das mit \enquote{/**} beginnt, dann folgen beliebige Zeichen und endet mit \enquote{*/}. Traditionell haben die Bezeichner der Tokens nur Großbuchstaben, sie müssen auf jeden Fall mit einem Großbuchstaben beginnen \cite[S. 3]{ANTLR:APredicated-<i>LLk</i>ParserGenerator}. 

Der nächste Schritt ist das eigentliche Parsen. Dabei werden die im vorherigen Schritt generierten Tokens genommen und in eine Baumstruktur umgewandelt, wobei hier der Kontext die entscheidende Rolle spielt. So kann eine vereinfachte Parser-Grammatik für Java einen Baumknoten definieren, der eine generelle Methode beschreibt. Eine Methode besteht nämlich aus Rückgabetyp, Bezeichner und Parameterliste. Die Parameterliste kann dann als eine Liste von Datentyp-Bezeichner-Paare aufgelöst werden. Der Rückgabetyp einer Methode ist allerdings anders zu verstehen als der Datentyp eines Parameters, da jeder Datentyp eines Parameters auch ein gültiger Rückgabetyp ist, was jedoch nicht umgekehrt der Fall ist; so ist \textit{}{void} ein gültiger Rückgabetyp, aber kein Datentyp. Listening ... zeigt ein typischer Ausschnitt aus einer Parser-Grammatik.
		\begin{figure} [htbp]
			\lstinputlisting
			[caption={Beispielhafte Syntax vom Parser},
			label={lst:parser_example},
			captionpos=b, basicstyle=\footnotesize, tabsize=2, showstringspaces=false,  numbers=left]
			{figures/parser_example.g4}
		\end{figure}
Die Parser-Grammatik ist ähnlich wie die Lexer-Grammatik, allerdings muss der Bezeichner vor dem Doppelpunkt(Regel) mit Kleinbuchstaben beginnen\cite[S. 3]{ANTLR:APredicated-<i>LLk</i>ParserGenerator}. 


Basierend auf diese Baumstruktur des Parsers kann eine Quellcodedatei analysiert werden und so alle relevanten Informationen gesammelt werden. Dies lässt sich relativ leicht mit dem Visitor-Pattern erledigen, da so nur die relevanten  Komponenten genauer betrachtet werden müsse und alle uninteressanten Komponenten automatisiert durchlaufen werden, bis eine relevante Komponente gefunden wird. Listening \ref{lst:visit_method_example} zeigt einen Ausschnitt aus dem Java-Parser um das Visitor-Pattern genauer zu erläutern.
		\begin{figure} [htbp]
			\lstinputlisting
			[caption={Codeauschnitt aus  Methoden-Visitor},
			label={lst:visit_method_example},
			captionpos=b,language=java, basicstyle=\footnotesize, tabsize=2, showstringspaces=false,  numbers=left]
			{figures/visit_method_example.java}
		\end{figure}
Das Code-Snippet zeigt den einen Ausschnitt vom Visitor für Methodendeklarationen. Hier ist die Baumstruktur leicht sichtbar. Alle Einzelbestandteile einer Methode wie z. B. Bezeichner, Rückgabetyp etc. sind Kindknoten des \textit{RuleContext} und können über die Methode \textit{getChild} abgefrufen werden. So werden sowohl der Bezeichner als auch der Rückgabetyp direkt als Text abgerufen. Theoretisch bestehen diese beiden Bestandteile auch aus weiteren Kindknoten, doch eine weitergehende Betrachtung ist nicht nötig, da nur die Bezeichnung als Zeichenkette benötigt wird. Andere Bestandteile wie die Methodenparameter sind jedoch komplexer, aus diesem Grund werden sie von separaten Visitors betrachtet.

Für die Programmiersprache Java steht bereits eine Grammatik, die auf Github unter der BSD-Lizenz angeboten wird ist, zur Verfügung\cite{antlr_grammar_github}, allerdings ignoriert sie die Kommentare. daher mussten einige Änderungen sowohl am Lexer als auch am Parser vorgenommen werden. Im Lexer werden standardmäßig alle Tokens in einem Kommentar in einen versteckten Kanal gespeichert, was dazu führt, dass diese Tokens vom Parser ignoriert werden. Daher wurde dieses Verhalten durch Definition eines neuen Tokens so geändert, dass Javadoc-Kommentare auch vom Parser verarbeitet werden können, aber mehrzeilige und einzeilige Kommentare weiterhin ignoriert werden. Einzeilige Kommentare sind hier nicht relevant, da sie kein Javadoc enthalten. Mehrzeilige Kommentare könnten theoretisch auch berücksichtigt werden, da einige Entwickler diese anstelle von Javadoc benutzen. Allerdings werden solche mehrzeilige Kommentare vor Komponenten nicht von Tools erkannt und haben daher einen geringeren, aber durchaus vorhandenen Nutzen \cite[S. 4]{HowDocumentationEvolvesoverTime}. Deshalb wurde entschieden, Komponenten, die zwar mit mehrzeiligen Kommentare aber nicht mit Javadoc dokumentiert sind, wie undokumentierte Komponenten zu betrachten. Für einen Entwickler sollte es so schnell möglich sein, solche nicht korrekt dokumentierten Komponenten zu identifizieren und deren mehrzeilige Kommentare in gültige Javadoc-Kommentare umzuwandeln und so die Qualität der Dokumentation zu erhöhen. Für andere Programmiersprachen können jedoch normale mehrzeilige wie strukturierte Kommentare betrachtet werden, wenn dies für sinnvollerer erachtet wird.

Mehr Änderungen mussten an der entsprechenden Parser-Datei \textit{JavaParser.g4} durchgeführt werden.  Tabelle \ref{tab:parser_changes} listet alle Änderungen an der Parserdatei auf;
\begin{table}[h]
    \centering
    \begin{tabular}{m{0.75cm}|m{4cm}|m{10cm}}
        Zeile & Änderung & Begründung \\
         \hline
        116 & Deklaration Kommentar & Hier wird ein mehrzeiliger Kommentar definiert, dies ist hier ein Alias für den Token \textit{JCOMMENT}\\
        \hline
        128 & \textit{comment} als möglicher Präfix in Klassenmember & Hier wird dem Parser mitgeteilt, dass ein Bestandteil einer Klasse wie z. B. eine Methode einen Javadoc-Kommentar besitzen kann\\
        \hline
        47 & \textit{comment} als möglicher Präfix vor Datentyp & Hier wird dem Parser mitgeteilt, dass ein Datentyp (Klasse, Schnittstelle etc. ) einen Javadoc-Kommentar haben kann \\
        \hline
        404 & Zulassung von Javadoc in Methoden & Da Javadoc-Kommentare an beliebigen Stellen auftauchen können, auch wenn es nicht empfohlen wird und keinen Mehrwert bietet, wird hier sichergestellt, dass solche Kommentare nicht zu Warnungen oder Fehler von ANTLR4 führen. Diese Javadoc-Kommentare werden nichtsdestotrotz später ignoriert.\\
        \hline
        34, 38& Zulassung von Kommentaren vor Paketdeklarationen und Imports & Hier werden Kommentare auch vor Paketdeklarationen und Import-Statements erlaubt, was vor allem bei Klassen mit Urheberrechtsangabe sinnvoll ist\\
        \hline
        105 & Zulassung von Kommentaren bei Enumerationen & Zwar werden Javadoc-Kommentare in Enumerationen mit diesem Tool nicht betrachtet, sie führen aber dennoch zu Warnungen und Fehlermeldungen. Daher werden sie hier zugelassen, aber später ignoriert. \\
        \hline
        82, 83 & Erzeugung eines separaten Knotens für \textit{Extends}- und \textit{Implements}-Deklarationen & In der originalen Version der Parserdatei wurde die Definition der Basisklasse bzw. der implementierten Schnittstellen direkt über die Tokens \textit{EXTENDS} bzw. \textit{IMPLEMENTS} gelöst. Dies wurde in einem neuen Knoten \textit{extendClass} bzw. \textit{implementInterfaces} ausgegliedert, um so das Parsing etwas zu vereinfachen.  \\
         \hline
    \end{tabular}
    \caption{Änderungen an der Parserdatei}
    \label{tab:parser_changes}
\end{table}

\subsection{Metriken anwenden}
Als nächster Schritt müssen auf die Informationen bezüglich der Dokumentation, die der Parser im vorherigen Schritt gesammelt hat, geeignet ausgewertet werden. Dazu können verschiedene Metriken verwenden, die im Späteren genauer erläutert werden. Eine Metrik analysiert genau eine Komponente und die dazugehörige Dokumentation. Jede Metrik gibt eine numerische Bewertung ab, die eine Aussage über die Dokumentationsqualität trifft. Hier wurde eine 0 für sehr schlechte bis nicht existente Dokumentation und 100 für eine exzellente Dokumentation gewählt, sodass die Bewertung sich als Prozent lesen lassen kann. Grundsätzlich ist aber auch ein anderer Wertebereich möglich. Eine Metrik kann sich aber auch dafür entscheiden, eine Komponente komplett zu ignorieren, sodass sie für diese Metrik als nicht existent betrachtet wird.

Zudem können Metriken Warn- oder Fehlermeldungen werfen, um den Anwender des Programms auf schlechte Dokumentation aufmerksam zu machen. Diese Fehlermeldung enthalten auch den Komponentennamen und die Zeilennummern, damit der Programmierer schnell auf die verwendete Komponente zugreifen kann.\bigskip
\subsection{Resultate der Metriken anwenden}
Da in einer Datei mehrere Komponenten durchaus Standard sind, müssen die Bewertungen jeder Komponente passend aggregiert werden. Dazu wurde ein \textit{MetricResultBuilder} entwickelt. Dieser kann durch die Methode \textit{processResult} das Resultat einer Metrik verarbeiten, indem es sie in einer Liste speichert. Wenn alle Metriken verarbeitet sind, muss daraus ein Gesamtresultat gebildet werden. Dies geschieht durch die Methode \textit{getAggregratedResult}. Dabei wird standardmäßig ein einfacher arithmetischer Mittelwert gebildet. Es sind auch aber auch andere \enquote{ResultBuilder} standardmäßig implementiert, sodass der Benutzer des Tools einen geeigneten Algorithmus auswählen kann. In den folgenden Unterabschnitten wird jeder Algorithmus kurz erläutert und dabei kurz auf die Vor- und Nachteile eingegangen. 

\subsubsection{Einfacher Mittelwert}
Der einfache arithmetische Mittelwert wird bereits durch die Klasse \textit{MetricResultBuilder} implementiert. Er summiert die Ergebnisse auf und teilt diese durch die Anzahl der Einzelresultate. Dieser Algorithmus ist einfach zu verstehen und berücksichtigt jedes Ergebnis gleichermaßen. Dies ist insbesondere für die Aggregierung der Ergebnisse der einzelnen Dateien sinnvoll, da jede Datei gleichbehandelt werden sollte und eine Gewichtung von möglicherweise Tausenden von Dateien nur schwer umsetzbar ist. Es sollte allerdings auch beachtet werden, dass der Mittelwert extreme Ausreißer berücksichtigt. Dies ist hier manchmal sinnvoll, da sehr schlechte oder sehr gute Dokumentation besser berücksichtigt wird und so ein verlässliches Gesamtergebnis geliefert wird.

\subsubsection{Median}
Der Median der Einzelresultate wird von der Klasse \textit{MedianResultBuilder} berechnet. Dabei werden die Einzelresultat nach ihrem Gewicht sortiert und dann das Resultat genommen, das in der sortierten Liste in der Mitte liegt. Bei einer geraden Anzahl an Elementen wird der Median aus dem Mittelwert der zwei infrage kommenden Ergebnissen gebildet. 

Der Median berücksichtigt extreme Ausreißer nicht und kann daher interessant sein, wenn ein allgemeines Bild von der Dokumentationsqualität gemacht werden soll. Es sollte aber beachtet werden, dass die Anwendung des Medians ein Sortiervorgang benötigt, der mindestens eine Komplexität von $O(n*log(n)$ hat. Bei der überschaubaren Anzahl an Metriken wird dies nicht so relevant sein, bei einer Anwendung des Medians auf vielen Dateien durchaus wohl.

\subsubsection{Gewichteter Mittelwert}
Der gewichtete Mittelwert ist in der Klasse \textit{WeightedMetricResultBuilder} implementiert. Dabei wird dem Konstrucktor dieser Klasse eine assoziative Liste übergeben, die einem Metrikobjekt ein Gewicht zuordnet.  Es wird dabei davon ausgegangen, dass jede Metrik nur von einer Instanz repräsentiert wird. Dies ist nur möglich, da die Metriken keinen inneren Zustand haben und nur mittels lokale Variablen und Parameter arbeiten. Jedes Einzelresultat speichert zudem den Erzeuger des Ergebnisses, sodass für jede Metrik ein Gewicht abgerufen werden kann. Die Gewichte müssen nicht normiert werden, da dies während der Berechnung implizit gemacht wird. Die Resultate jeder Metrik werden multipliziert mit dessen Gewicht aufsummiert und dann durch die Summe aller Gewichte geteilt. 

Dieser Algorithmus ermöglicht es, bestimmte Metriken zu bevorzugen bzw. zu benachteiligen. Dies ist sinnvoll, da nicht jede Metrik immer ein aussagekräftiges liefert und bestimmte Metriken je nach Situation ein besseres Bild über die Dokumentationsqualität liefert. Allerdings ist auch zu beachten, dass die Wahl der Gewichte nicht trivial ist und ein Vergleich von Ergebnissen, die verschiedene Gewichte verwenden, nur selten sinnvoll ist.

\subsubsection{Gewichteter Median}
Der gewichtete Median wurde leicht abweichend nach \cite[S. 37]{YAGER199835} implementiert. Dabei wird zunächst die Summe der Gewichte berechnet und die Resultate nach ihrem Gewicht sortiert. Anschließend werden die sortierten Resultate und ihre Gewichte so lange aufsummiert, bis diese temporäre Summe die Hälfte der Gesamtsumme überschreitet. Das Metrikergebnis, bei der diese Bedingung zutrifft, ist das gesuchte Gesamtergebnis. Die Vor- und Nachteile dieses Algorithmus entsprechen in etwa den Vor- und Nachteilen des gewichteten Mittelwerts und des Medians. So muss auch hier eine Sortierung durchgeführt werden und die Wahl der richtigen Gewichte ist nicht trivial. 
\section{Konfiguration des Tools}
\begin{table}[h]
    \centering
    \begin{tabular}{m{4cm}|m{11cm}}
    \hline
        Parameter & Beschreibung  \\
        \hline
        include & Alle Dateien, die bei der Bewertung der Dokumentationsqualität berücksichtigt werden müssen\\
        \hline
        Exclude & Teilmenge von include, enthält Dateien, die nicht weiter betrachtet werden müssen\\
        \hline
        global\_threshold & Mindestwert der Bewertung ,der erreicht werden muss, sonst wird die Dokumentationsqualität nicht akzeptiert
    \end{tabular}
    \caption{Parameter zur Konfiguration des Tools}
    \label{tab:tool_javadoc_conf}
\end{table}
\todo{Add more if tool is more developed}
\subsection{Einbindung in GitHub Actions}
