\section{Metriken}
\subsection{Einleitung}
Im Folgenden werden die Metriken präsentiert, die in der finalen Version des \textit{JavadocEvaluator} implementiert worden sind. Dabei werden pro Metrik einige Vor- und Nachteile genannt, um den zukünftigen Anwender eine Entscheidungshilfe zu bieten. 
\subsection{Metrik: Anteil dokumentierter Komponenten an allen Komponente}
Bei dieser Metrik werden dokumentierte Komponenten mit hundert Punkten bewertet, während undokumentierte Komponenten null Punkte erhalten. Dabei ist es irrelevant, wie gut die Dokumentation ist. Selbst  die leere Dokumentation \textbf{/***/} würde einen Rating von 100 erhalten, da hier eine weitergehende Differenzierung zwar umsetzbar ist, aber dafür eine davon abgeleitete Metrik besser geeignet wäre. Vorteilhaft an dieser Metrik ist die Einfachheit der Implementierung. Außerdem ist sie als grober Indikator gut geeignet, die Qualität der Dokumentation abzuschätzen. Ein Nachteil an dieser Metrik ist, dass sie auch sinnlose Kommentare, die leer oder nichts mit der dokumentierten Komponente zu tun haben, mit der vollen Punktzahl bewertet werden. Außerdem bestehen moderne Softwareprojekte aus vielen verschiedenen Komponenten, wobei einige Komponenten nur an einigen wenigen Stellen verwendet werden und daher nur selten dokumentiert wurden. Aus diesen Grund würde diese Metrik die \enquote{wahre} Dokumentationsqualität unterschätzen.

\subsection{Metrik: Anteil dokumentierter öffentlicher Komponenten an allen öffentlichen Komponenten }
Diese Metrik basiert auf der vorherigen Metrik und verwendet sie sogar intern. Im Unterschied zu dieser werden hier nur öffentliche Komponenten betrachtet. Dies bedeutet, dass alle nichtöffentlichen Komponenten so betrachtet werden, als wären sie nicht existent. Die öffentlichen Komponenten dagegen werden wie oben mit hundert bzw. null Punkten bewertet. Dies hat den Vorteil, dass nur die Komponenten, die wahrscheinlich von anderen Komponenten verwendet werden müssen und daher auch gut verstanden werden müssen, bei der Bewertung der Qualität der Dokumentation berücksichtigt werden. So hat der Entwickler einen guten Überblick über die Stellen, die nach gängiger Erfahrung unbedingt besser dokumentiert werden müssen. Auf der anderen Seite werden sinnlose Kommentare weiterhin akzeptiert und nichtöffentliche Komponenten werden ohne Berücksichtigung ihrer Wichtigkeit ignoriert, obwohl auch hier eine gute Dokumentation Fehler vermeiden kann und zu einer besseren Softwarequalität führen kann. 