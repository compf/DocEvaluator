\label{sec:introduction}

Ein wichtiger Bestandteil der Softwareentwicklung von heute ist die Softwaredokumentation. Dies liegt unter anderem daran, dass die Größe von Softwareprojekten steigt, sodass die Entwickler schnell den Überblick über das Projekt verlieren können und daher zusätzliche Informationen neben dem Code benötigen \cite[S.~1]{StaticAnalysis:AnIntroduction:TheFundamentalChallengeofSoftwareEngineeringisOneofComplexity.}. Nichtsdestotrotz wird die Softwaredokumentation von Entwicklern oft vernachlässigt \cite[S.~83]{Qualityanalysisofsourcecodecomments}.  Die Gründe für schlechte Dokumentation sind vielfältig. Das Schreiben der Dokumentation wird oft als mühevoll empfunden und erfordert Fähigkeiten, die ein Programmierer nicht zwangsläufig besitzt \cite[S.~70]{AutomaticQualityAssessmentofSourceCodeComments:TheJavadocMiner} \cite[S.~593]{Softwareengineeringandsoftwaredocumentation:aunifiedlongcourse}.  

Weitere Studien verdeutlichen die Problematik der mangelhaften Softwaredokumentation. So belegt eine Umfrage aus dem Jahr 2002 mit 48 Teilnehmern  beispielsweise, dass die Dokumentation  bei Änderungen am System  nur mit Verzögerung angepasst wird. Knapp 70~\% der Teilnehmer stimmen der Aussage zu, dass die Dokumentation immer veraltet ist.   \cite[S.~28-29]{TheRelevanceofSoftwareDocumentationToolsandTechnologies:ASurvey}

Eine weitere Studie  \cite[S.1199-1208]{SoftwareDocumentationIssuesUnveiled} aus dem Jahr 2019 verdeutlicht viele Aspekte aus der vorgenannten Umfrage. Es wurden dabei Daten aus Stack Overflow, GitHub-Issues, Pull-Requests und Mailing-Listen automatisiert heruntergeladen und dann von den Autoren analysiert, ob und inwieweit mit mangelhafter Softwaredokumentation zu tun haben.  Die Studie belegt, dass von 824 Problemen, die etwas mit dem Thema \enquote{Softwaredokumentation} zu tun haben, 485 sich auf den Inhalt der Dokumentation beziehen (wie z.~B. unvollständige, veraltete oder sogar inkorrekte Dokumentation). Bei 255 Einträgen gab es Probleme mit der Struktur der Dokumentation, sodass beispielsweise Informationen schlecht auffindbar sind oder nicht gut verständlich sind.


Eine andere Umfrage aus dem Jahr 2014 mit 88 Teilnehmern zeigt, dass eine automatisierte Überprüfung der Dokumentationsqualität von knapp der Hälfte der befragten Entwickler gewünscht wird. Die Autoren der Studie sehen dies als Zeichen dafür, dass ein grundsätzliches Bedürfnis zur automatisierten Bewertung von Dokumentationen besteht und daher weitere Studien notwendig sind. \cite[S.~340]{TheValueofSoftwareDocumentationQuality}

Die mangelhafte Dokumentation führt dazu, dass nicht nur nachfolgende Entwickler Probleme mit dem Codeverständnis haben, sondern auch Entwickler eines Moduls nach einer längeren Pause Zeit aufbringen müssen, um den Code wieder zu verstehen \cite[S.~511]{vestdam}.  Auch für Kunden/Auftraggeber ist eine gute Dokumentation wichtig, da gut dokumentierte Software tendenziell besser wartbar ist und somit mehr Nutzen bringt \cite[S.~83]{Qualityanalysisofsourcecodecomments} \cite[S.~1]{SoftwareDocumentationManagementIssuesandPractices:ASurvey}.



\section{Zielsetzung}
Aufgrund der Relevanz von gut dokumentierter Software ist eine regelmäßige Rückmeldung über die Dokumentation von hoher Bedeutung. Spezielle Metriken, die eine numerische Auskunft über die Qualität der Softwaredokumentation liefern, sind eine Möglichkeit diese Rückmeldung zu geben. Diese Metriken liefern dem Programmierer eine Einschätzung darüber, ob die Softwaredokumentation ausreichend ist oder eine Verbesserung sinnvoll wäre. Die Qualität der Softwaredokumentation kann auf unterschiedliche Art und Weise bewertet werden. So kann beispielsweise die bloße Existenz einer Dokumentation geprüft werden oder aber auch die Verständlichkeit der Dokumentation bewertet werden, daher kann es sinnvoll sein, mehrere Metriken zu verwenden \cite[S.~29]{Pfleeger2005UsingMM}. Damit ein Entwickler einen Gesamtüberblick über die Dokumentationsqualität erhält, können diese Metriken kombiniert werden, um eine einzelne numerische Bewertung der Qualität der Dokumentation zu erhalten. 
Dabei ist es auch ratsam, die Metriken zu gewichten oder eine andere Methode zur Kombination der Metrikergebnisse zu benutzen, weil nicht jede Metrik die gleiche Zuverlässigkeit und Relevanz besitzt \cite[S.~1117ff.]{Softwarequalitymetricsaggregationinindustry}.

Damit das Feedback über die Softwaredokumentation auch wahrgenommen wird, sollte die Qualität regelmäßig  überprüft werden. Dies kann automatisiert im \ac{CI/CD}-Prozess erfolgen, bei dem Software kontinuierlich getestet und für den Release (z.~dt. Veröffentlichung) vorbereitet werden kann. Durch CI/CD können Unternehmen effizienter und besser Software entwickeln. So konnte das Unternehmen \enquote{ING NL} die gelieferten Function-Points vervierfachen und die Kosten für einen Function-Point auf einem Drittel reduzieren \cite[S.~520]{Vassallo2016}.

\hfill

Basierend auf diesen Überlegungen soll ein Tool (z.~dt. Werkzeug) entwickelt werden. Dieses Tool soll ein gegebenes Software-Projekt analysieren und eine numerische Bewertung abgeben, die eine heuristische Aussage über die Qualität der Softwaredokumentation trifft.  Dabei soll das Tool primär für Javadoc und Java bis Version 8 konzipiert werden, allerdings soll während der Entwicklung auch darauf geachtet werden, dass eine Portierung auf eine andere Programmiersprache ermöglicht wird und die Bewertung der Dokumentation unabhängig von der Programmiersprache funktioniert. Außerdem wird zur Vereinfachung nur englischsprachige Dokumentationen betrachtet. Komplexe Natural-Language-Processing-Metriken sollen dabei außer Acht gelassen werden. 

Dabei sollte es nicht unbedingt das Ziel sein, dass jede Komponente dokumentiert ist, sondern dass die wichtigen Komponenten eine gute Dokumentationsqualität haben und somit die Wartung vereinfacht wird. Als Komponente im Sinne dieser Bachelorarbeit werden dabei Klassen, Schnittstellen, Methoden und Felder verstanden. 

Dieses Tool soll anschließend in dem \ac{CI/CD}-Prozess eingebunden werden, sodass die Dokumentationsqualität kontinuierlich geprüft werden kann. Als \ac{CI/CD}-Plattform soll dabei \enquote{GitHub Action} \cite{GithubActions} verwendet werden, da GitHub von der Mehrzahl der Entwickler und großen Unternehmen verwendet wird \cite{github_popular}. Mittels GitHub Action soll das Tool bei einer sehr schlechten Dokumentationsqualität den Entwickler auf diesen Umstand hinweisen, indem beispielsweise ein Merge (z.~dt. Verschmelzung) in GitHub verhindert wird. Auch bei einer deutlichen inkrementellen Verschlechterung der Qualität soll der Entwickler informiert werden, um so eine ausreichende Qualität der Dokumentation sicherzustellen. 

Ein Forschungsziel dieser Bachelorarbeit ist es zu prüfen, wie das Programm konzipiert werden muss, um mehrere Programmiersprachen zu unterstützen. Ein weiteres Ziel der Arbeit beschäftigt sich mit der Frage, wie die Ergebnisse der Metriken kombiniert werden können, um eine präzise Aussage über die Gesamtqualität der Dokumentation eines Softwareprojektes zu erhalten. Die Konzeption einer Architektur, mit der weitere Metriken hinzugefügt werden können und der Nutzer des Tools auswählen kann, welche Metriken bei der Bewertung des Dokumentationsqualität berücksichtigt werden sollen, soll ebenfalls als Forschungsziel untersucht werden. Zuletzt soll als Forschungsfrage diskutiert werden, welche Metriken eine heuristische Aussage über die Qualität der Dokumentation treffen können. 


\section{Gliederung}
In Kapitel \ref{sec:background} werden die wichtigen Grundlagen über die Themen dieser Bachelorarbeit erläutert. Dazu  wird zunächst der Begriff Softwaredokumentation definiert und ein Bezug zu Code-Smells hergestellt. Mittels Javadoc wird dann erläutert, wie Software dokumentiert werden kann. Anschließend wird eine Einführung in GitHub Actions gegeben. Zuletzt werden einige wissenschaftliche Arbeiten mit vergleichbaren Zielsetzungen präsentiert und Tools vorgestellt, die ebenfalls die Qualität der Softwaredokumentation bewerten können.

In Kapitel \ref{chapter_conception} werden die Fragestellungen besprochen, die sich beim Design des Tools entwickelt haben. Dazu gehören die notwendigen Objekte und ihre Interaktion untereinander und wie von einer losen Ansammlung von Dateien zu einer Bewertung der Softwaredokumentation gelangt werden kann.

In Kapitel \ref{chapter:program} wird anschließend erläutert, wie aus dieser Konzeption ein vollständiges Programm entwickelt wurde. Dazu wird eine Einführung in ANTLR4 gegeben, das für das Parsing der Quellcodedateien in Java verwendet wird. Zudem wird erläutert, wie das Programm in GitHub Action eingebunden werden kann. Im Anschluss daran wird ein Überblick über die implementierten Metriken mit ihren Vor- und Nachteilen gegeben. Außerdem werden die Algorithmen bzw. Verfahren erläutert, um die Ergebnisse der einzelnen Metriken zu einem Gesamtergebnis zu aggregieren. 

In Kapitel \ref{sec:evaluation} wird das Programm dann mit ähnlichen Tools verglichen, indem beispielhafte Java-Projekte aus GitHub mit allen Programmen analysiert werden und die Geschwindigkeit und die Qualität jedes Programmes ermittelt wird. 

Im abschließenden Kapitel wird der Inhalt der Arbeit zusammengefasst und ein Fazit gezogen. Es werden offengebliebene Fragen beleuchtet und ein Ausblick gegeben, welche Möglichkeiten zur Verbesserung des Tools sinnvoll wären. 
