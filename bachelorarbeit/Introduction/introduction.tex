\label{sec:introduction}

Ein wichtiger Bestandteil der Softwareentwicklung von heute ist die Softwaredokumentation. Dies liegt unter anderem daran, dass die Größe von Softwareprojekten steigt, so dass kein Entwickler das gesamte Programm im Überblick hat und daher zusätzliche Informationen neben dem Code benötigt \cite[S. 1]{StaticAnalysis:AnIntroduction:TheFundamentalChallengeofSoftwareEngineeringisOneofComplexity.}. Nichtsdestotrotz wird die Softwaredokumentation von Entwicklern oft vernachlässigt \cite[S. 83]{Qualityanalysisofsourcecodecomments}.  Die Gründe für schlechte Dokumentation sind vielfältig. Das Schreiben der Dokumentation wird oft als mühevoll empfunden und erfordert Fähigkeiten, die ein Programmierer nichts zwangsläufig hat\cite[S. 70]{AutomaticQualityAssessmentofSourceCodeComments:TheJavadocMiner} \cite[S. 593]{Softwareengineeringandsoftwaredocumentation:aunifiedlongcourse}.  

Die mangelhafte Dokumentation führt dazu, dass nicht nur nachfolgende Entwickler Probleme mit dem Codeverständnis haben, sondern auch der Entwickler eines Moduls nach einer längeren Pause Zeit aufbringen muss, um den Code wieder zu verstehen. Auch für Kunden/Auftraggeber ist eine gute Dokumentation wichtig, da gut dokumentierte Software tendenziell besser wartbar ist und somit mehr Nutzen bringt \cite[S. 83]{Qualityanalysisofsourcecodecomments}\cite[S. 1]{SoftwareDocumentationManagementIssuesandPractices:ASurvey}.




