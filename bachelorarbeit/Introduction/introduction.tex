\label{sec:introduction}

Ein wichtiger Bestandteil der Softwareentwicklung von heute ist die Softwaredokumentation. Dies liegt unter anderem daran, dass die Größe von Softwareprojekten steigt, sodass kein Entwickler das gesamte Programm im Überblick hat und daher zusätzliche Informationen neben dem Code benötigt \cite[S. 1]{StaticAnalysis:AnIntroduction:TheFundamentalChallengeofSoftwareEngineeringisOneofComplexity.}. Nichtsdestotrotz wird die Softwaredokumentation von Entwicklern oft vernachlässigt \cite[S. 83]{Qualityanalysisofsourcecodecomments}.  Die Gründe für schlechte Dokumentation sind vielfältig. Das Schreiben der Dokumentation wird oft als mühevoll empfunden und erfordert Fähigkeiten, die ein Programmierer nichts zwangsläufig hat\cite[S. 70]{AutomaticQualityAssessmentofSourceCodeComments:TheJavadocMiner} \cite[S. 593]{Softwareengineeringandsoftwaredocumentation:aunifiedlongcourse}.  

Die mangelhafte Dokumentation führt dazu, dass nicht nur nachfolgende Entwickler Probleme mit dem Codeverständnis haben, sondern auch der Entwickler eines Moduls nach einer längeren Pause Zeit aufbringen muss, um den Code wieder zu verstehen. \cite[S. 511]{vestdam} Auch für Kunden/Auftraggeber ist eine gute Dokumentation wichtig, da gut dokumentierte Software tendenziell besser wartbar ist und somit mehr Nutzen bringt \cite[S. 83]{Qualityanalysisofsourcecodecomments}\cite[S. 1]{SoftwareDocumentationManagementIssuesandPractices:ASurvey}.


Dass in vielen Fällen die Softwaredokumentation vernachlässigt wird, ist durch viele Studien belegt. Eine Umfrage aus dem Jahr 2002 mit 48 Teilnehmern belegt beispielsweise, dass Anforderungs- oder Spezifikationsdokumente nur selten bei Änderungen am Quellcode angepasst werden. Außerdem finden 54\% der befragten Entwickler  Textverarbeitungsprogramme wie Word für hilfreich, die nicht für Dokumentationszwecke entwickelt wurden.Diese Programme sind leicht zu bedienen und sehr flexibel, aber nicht effizient  \cite[S. 28-29]{TheRelevanceofSoftwareDocumentationToolsandTechnologies:ASurvey}. 

Eine weitere Studie aus dem Jahr 2019 verdeutlicht viele Aspekte aus der vorgenannten Umfrage. Es wurden dabei Daten aus Stack Overflow, GitHub Issues und Pull Requests und Mailing-Listen automatisiert heruntergeladen und dann von den Autoren analysiert, ob und inwieweit mit mangelhafter Softwaredokumentation zu tun haben.  Die Studie liefert klare Indizien dafür, dass die Softwaredokumentation in vielen Fällen nicht komplett, nicht auf dem neuesten Stand oder sogar nicht korrekt ist. Des Weiteren ist die Softwaredokumentation nicht gut nutzbar oder schlecht lesbar, sodass der Vorteil verloren geht\cite[S.1201 -1204]{SoftwareDocumentationIssuesUnveiled}.







\section{Zielsetzung}
Aus diesen Gründen ist eine regelmäßige Rückmeldung über die Dokumentation von hoher Bedeutung. Eine Möglichkeit, dieses Feedback zu geben, sind spezielle Metriken, welche eine numerische Auskunft über die Qualität der Softwaredokumentation liefern. Diese Metriken liefern dem Programmierer eine gute Einschätzung, ob die Softwaredokumentation ausreichend ist oder eine Verbesserung sinnvoll wäre. Da es im Bereich der Softwaredokumentation, verschiedene Metriken gibt, ist es sinnvoll mehrere Metriken zu verwenden. Die Ergebnisse aller Metriken können dann kombiniert werden. Dabei ist es auch ratsam, die Metriken zu gewichten oder eine andere Methode zur Kombination der Metrikergebnisse zu benutzen, weil nicht jede Metrik die gleiche Zuverlässigkeit und Relevanz besitzt \cite[S. 1117ff.]{Softwarequalitymetricsaggregationinindustry}.

Damit das Feedback über die Softwaredokumentation auch wahrgenommen wird, sollte die Qualität regelmäßig  überprüft werden. Dies kann automatisiert in \ac{CI/CD}-Prozess erfolgen, bei dem Software kontinuierlich getestet und für den Release (z.~dt. Veröffentlichung) vorbereitet werden kann. Durch CI/CD können Unternehmen effizienter und besser Software entwickeln. So konnte die ING NL die gelieferten Function-Points vervierfachen und die Kosten für einen Function-Point auf einen Drittel reduzieren \cite[S. 520]{Vassallo2016}.

Basierend auf diesen Überlegungen soll ein Tool(z.dt. Werkzeug) entwickelt werden. Dieses Tool soll ein gegebenes Software-Projekt analysieren und eine numerische Bewertung darüber abgeben, die eine heuristische Aussage über die Qualität der Softwaredokumentation trifft.  Dabei soll das Tool primär für Javadoc konzipiert werden, allerdings soll während der Entwicklung auch darauf geachtet werden, dass eine Portierung auf eine andere Programmiersprache möglichst einfach wird und die Bewertung der Dokumentation unabhängig von der Programmiersprache funktioniert.Außerdem soll zur Vereinfachung sollen dabei nur englischsprachige Dokumentation betrachtet werden.

Ziel dabei soll es nicht sein, dass jede Komponente, sondern dass die wichtigen Komponenten eine gute Dokumentationsqualität haben und somit die Wartung vereinfacht wird.  Als Komponente im Sinne dieser Bachelorarbeit werden dabei Klassen, Schnittstellen, Methoden und Felder verstanden. 



\section{Gliederung}
Zur Umsetzung der Bachelorarbeit wird zunächst ein Einblick in das Thema Code-Smell  gegeben, da schlechte Softwaredokumentation als Code-Smell angesehen werden kann. Anschließend muss der Begriff der Softwaredokumentation genauer definiert werden, da der Begriff so noch sehr unpräzise ist. Zudem wird eine Einführung in Javadoc gegeben, da dieses Tool am Ende der Bachelorarbeit hauptsächlich Java und Javadoc verarbeiten soll.
Außerdem werden einige verwandte Programme vorgestellt, die ebenfalls die Qualität der Softwaredokumentation analysieren können. Des Weiteren werden einige wissenschaftliche Arbeiten zusammengefasst, die sich ebenfalls mit der Analyse von Softwaredokumentationen beschäftigen.

Um das Programm in dem \ac{CI/CD}-Prozess einzubinden, wird \enquote{GitHub Actions} verwendet. Dieser Service der Plattform GitHub \footnote{\href{https://github.com/}{Github Website (besucht am 07.01.2022)}} ermöglicht es, eigene Programme (oder Programme aus einer großen Auswahl von anderen Programmierern) auf den Quelltext auszuführen.  Diese Programme können dann Fehlermeldungen werfen, wenn es irgendwelche Probleme mit dem Quelltext gibt, sodass der Entwickler darauf aufmerksam wird und die Probleme lösen kann. Mit dieser Plattform ist damit möglich, die Qualität der Softwaredokumentation regelmäßig zu prüfen und bei einer deutlichen Unterschreitung den Entwickler durch Fehlermeldungen zu warnen.  

Danach wird in Kapitel \ref{chapter_conception} die Konzeption des Programms erläutert. Hierzu wird erläutert, welche Designentscheidungen getroffen werden müssen. Dazu werden die wichtigen Klasse und ihre Beziehungen zueinander erklärt. Außerdem werden die eventuelle Probleme und Fragestellungen beschrieben, die überwunden werden müssen, um von einer Ansammlung an Quellcodedateien zu einer Bewertung der Dokumentationsqualität zu gelangen.  Dazu gehört die Traversierung aller relevanten Dateien, eine allgemeine Struktur zur Repräsentation der verschiedenen Komponenten einer objektorientierten Programmiersprache und das Vorgehen zur Ermittlung der Dokumentationsqualität

In Kapitel \ref{chapter:program} wird anschließend erläutert, wie aus der abstrakten Konzeption ein vollständiges Programm entwickelt wurde, das in GitHub Action verwendet werden kann. Dazu wird gezeigt, wie das Programm intern mit den verschiedenen Komponenten aus der Konzeption interagiert, um das Ziel dieser Bachelorarbeit zu erfüllen. Es wird eine Einführung in ANTLR4 gegeben, mit dem Programmiersprachen geparst werden können. Zudem wird beschrieben, wie dann die ANTLR4-Bibliothek im Quellcode benutzt werden kann. Anschließend wird erläutert, wie das Tool auf die unterschiedlichen Bedürfnisse angepasst werden kann. Danach wird kurz dokumentiert, wie das Tool die Qualität der Dokumentation zwischenspeichern kann, um bei einem erneuten Programmstart die aktuelle Qualität mit der letzten Qualität vergleichen zu können. Zuletzt wird erläutert wie das Programm in GitHub Action eingebunden werden kann.   

In Anschluss daran werden alle implementierten Metriken erläutert. Für jede implementierte Metrik werden Hintergrundinformationen erläutert. Außerdem werden einige Details der Implementierung beschrieben und die Vor- und Nachteile der jeweiligen Metrik erklärt, um den Anwender bei der Wahl der richtigen Metriken zu helfen. Auf der gleichen Art und Weise werden einige Metriken beschrieben, die in der wissenschaftlichen Literatur oder in anderen ähnlichen Tools unberücksichtigt sind, aber durchaus interessant sein könnten. 

Anschließend soll das Tool mit anderen Tools, die ebenfalls die Dokumentationsqualität bewerten können, verglichen werden. Dazu werden ausgewählte Java-Projekte aus GitHub mit allen Tools analysiert und die Ergebnisse der Tools verglichen. 







