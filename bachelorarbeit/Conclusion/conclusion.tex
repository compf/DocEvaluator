In diesem Kapitel wird ein Ausblick gegeben und ein Fazit gezogen. Dazu wird erläutert, wie das Tool ergänzt werden kann und welche Probleme bei einer Erweiterung auftreten können
\section{Erweiterung des Tools}
Es gibt viele Ideen, um weitere Funktionen für den \textit{DocEvaluator} zu implementieren, dies es wegen der begrenzten Bearbeitungszeitraums allerdings nicht in der Abgabeversion geschafft haben. Diese Vorschläge und Ideen werden nachfolgend erläutert.
\subsubsection{Weitere Programmiersprachen}
In der ausgelieferten Version unterstützt das Tool nur Java, es wurde aber prinzipiell so konzipiert, dass eine Erweiterung auf weitere Programmiersprachen möglich ist. Dazu muss eine geeignete Parser-Klasse geschrieben werden, die eine Quellcodedatei in das in Kapitel \ref{chapter_conception} beschriebene Format überführt. 

Da sich die Programmiersprachen in einigen Punkten besondere Funktionen haben, müssen diese geeignet abstrahiert werden. Beispielsweise müssen Eigenschaften in C\# in einer geeigneten Unterklasse von \textit{Component} dargestellt werden. Dazu kann eine neue abgeleitete Klasse erstellt werden. Alternativ kann die Klasse \textit{SingleMemberComponent}  benutzt werden, da sich eine C\#-Eigenschaft von der Benutzung sich kaum von einem gewöhnlichen Feld unterscheidet.

Außerdem muss dann für die Programmiersprache die sprachspezifischen Hilfsmethoden passend überschrieben werden.

\subsubsection{Unterstützung mehrere Programmiersprachen}
Eine weitere Idee ist es, mehrere Programmiersprachen zu unterstützt, sodass  beispielsweise die Dokumentationsqualität von einen gemixten Java- und Python-Projekt bewertet werden kann. Dazu kann je nach Dateiendung ein passender Parser aufgerufen werden und dann die sprachunabhängigen Metriken angewendet werden.

\subsubsection{Unterstützung weitere natürlicher Sprachen}
In der ausgelieferten Fassung unterstützt das Tool nur Englisch als Sprache. Beispielsweise sind alle \ac{NLP}-Metriken sehr auf die englische Sprache bezogen. Eine Erweiterung auf andere Sprachen könnte in Erwägung gezogen werden. 

\subsubsection{Unterstützung von anderen CI/CD-Plattformen}
Neben GitHub bietet auch GitLab eine Plattform zur automatisierten Ausführung von Programmcode bei bestimmten Ereignissen an. Da GitHub nicht die einzige Plattform zum Arbeiten mit Softwareprojekten ist, wäre eine Erweiterung auf andere Plattformen sinnvoll. Dabei könnte auch eine Abstrahierung geprüft werden, um eine Portierung auf andere \ac{CI/CD}-Plattformen so einfach wie möglich zu machen. 






